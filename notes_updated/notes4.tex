In Notes 1 the true joint distribution of $(u,f)$ given by (\ref{trueJoint}) gives us that the ``true" distribution of the solution $u$ to the problem (\ref{standard_conduct}) is $\nu_{1}=\mathcal{N}(\mathcal{L}^{-1}\bar{f},\mathcal{L}^{-1}K(\mathcal{L}^{-1})^{*})$. Meanwhile, in Notes 2 we worked out that ``averaged" posterior for $u$ (for the FEM prior) is $\nu_{2}=\mathcal{N}(\Phi^{*}M^{-1}\bar{F},\Phi^{*}M^{-1}K_{\mathcal{I}}M^{-1}\Phi)$ (see (\ref{average_posterior_FEM_prior})). In this section we will consider quantifying how close these two distributions $\nu_{1},\nu_{2}$ are. This will be achieved by obtaining an upperbound for the Wasserstein distance between $\nu_{1},\nu_{2}$ using the connection between the Wasserstein distance between Gaussian measures and the Procrustes Metric on covariance operators (see \cite{masarotto2019procrustes}). We start by first giving the definition of the Wasserstein distance between two probability measures $\mu,\nu$ on $\mathcal{H}$, $W(\mu,\nu)$:
\begin{equation*}
    W^{2}(\mu, \nu)=\inf _{\pi \in \Gamma(\mu, \nu)} \int_{\mathcal{H} \times \mathcal{H}}\|x-y\|^{2} \mathrm{d} \pi(x, y)
\end{equation*}
where $\Gamma(\mu,\nu)$ is the set of couplings of $\mu$ and $\nu$, i.e: \\ $$\Gamma(\mu,\nu)=\{\text{Borel probability measures } \pi \text{ on } \mathcal{H}\times\mathcal{H}|\pi(E\times\mathcal{H})=\mu(E) \text{ and } \pi(\mathcal{H}\times F)=\nu(F) \text{ for all Borel } E,F\subset\mathcal{H}\}$$
When $\mu$ and $\nu$ are both Gaussian measures an explicit expression can be obtained for the Wasserstein distance. Suppose $\mu=\mathcal{N}(m_1,\Sigma_1)$ and $\nu=\mathcal{N}(m_2,\Sigma_2)$. One has (see \cite{masarotto2019procrustes}),
\begin{equation*}
    W^{2}(\mu, \nu)=\left\|m_{1}-m_{2}\right\|^{2}_{\mathcal{H}}+\operatorname{tr}\left(\Sigma_{1}\right)+\operatorname{tr}\left(\Sigma_{2}\right)-2 \operatorname{tr} \sqrt{\Sigma_{1}^{1 / 2} \Sigma_{2} \Sigma_{1}^{1 / 2}}
\end{equation*}
This formula is true in both the finite and infinite dimensional cases. The term $\operatorname{tr} \sqrt{\Sigma_{1}^{1 / 2} \Sigma_{2} \Sigma_{1}^{1 / 2}}$ is difficult to analyse in our situation. As such we will make use of Proposition 3 from \cite{masarotto2019procrustes} which states:
\begin{proposition}
    The Procrustes distance between two trace-class operators $\Sigma_{1}$ and $\Sigma_{2}$ on $\mathcal{H}$ coincides with the Wasserstein distance between two second-order Gaussian processes $\mathcal{N}(0,\Sigma_{1})$ and $\mathcal{N}(0,\Sigma_2)$ on $\mathcal{H}$,
    $$\Pi(\Sigma_{1},\Sigma_{2}):=\inf_{R:R^{*}R=I}\|\Sigma_{1}^{1/2}-R\Sigma_{2}^{1/2}\|_{2}=W(\mathcal{N}(0,\Sigma_{1}),\mathcal{N}(0,\Sigma_{2}))$$
    where $\|\cdot\|_{2}$ is the Hilbert-Schmidt norm defined by $\|A\|_{2}=\sqrt{\operatorname{tr}(A^{*}A)}$.
\end{proposition}
Using this result one can obtain a simple upperbound on the Wasserstein distance by choosing $R=I$ in the infimum:
\begin{equation*}
    W(\mathcal{N}(0,\Sigma_1),\mathcal{N}(0,\Sigma_2))=\inf_{R:R^{*}R=I}\|\Sigma_{1}^{1/2}-R\Sigma_{2}^{1/2}\|_{2} \leq \|\Sigma_{1}^{1/2}-\Sigma_{2}^{1/2}\|_{2}
\end{equation*}
Now since in our case we have un-centered Gaussian measures $\nu_{1},\nu_{2}$ we must first link the Wasserstein distance between $\nu_1,\nu_2$ to the Wasserstein distance of the centred measures $\nu_{1}^{*},\nu_{2}^{*}$ using a general result mentioned in \cite{cuesta1996lower}:
\begin{equation*}
    W^2(\nu_{1},\nu_{2})=\|m_1-m_2\|^{2}+W^{2}(\nu_{1}^*,\nu_{2}^{*})
\end{equation*}
\textit{Note: here} $m_1,m_2$ \textit{are the means of} $\nu_{1},\nu_{2}$ \textit{respectively}. \\

For our particular case, we will consider the underlying Hilbert space to be $\mathcal{H}=L^{2}(\Omega)$. As such the norm for the difference in means is the $L^{2}$ norm. However, note that the solution $u$ to our boundary value problem lies in $H^{1}_{0}(\Omega)$, a subspace of $L^{2}(\Omega)$. %Since we are interested in seeking an upper bound on the Wasserstein distance we can focus on the $|\cdot|_{H^{1}_{0}(\Omega)}$ norm\footnote{This is a semi-norm on $H^{1}(\Omega)$.} on $H_{0}^{1}(\Omega)$: $|m_1-m_2|_{H^1(\Omega)}^{2}=\|\nabla(m_1-m_2)\|_{L^2(\Omega)}^{2}$. Since this norm is equivalent to the usual $H^1(\Omega)$ norm on $H_{0}^{1}(\Omega)$, which is an upper bound for the $L^{2}$ norm, we can go about finding an upper bound for the $|\cdot|_{H_{0}^{1}(\Omega)}$ norm.
If we now denote the covariance operators of $\nu_{1},\nu_{2}$ as $\Sigma_{1},\Sigma_{2}$ respectively we have:
\begin{equation*}
    W^{2}(\nu_1,\nu_2)=\|m_1-m_2\|_{L^{2}(\Omega)}^{2}+W^{2}(\mathcal{N}(0,\Sigma_1),\mathcal{N}(0,\Sigma_2))
\end{equation*}
We now go about obtaining an upperbound on each of these two terms. We want to control each term by the FEM mesh size $h$ (which is inversely proportional to the number of finite elements $J$). For brevity we will assume now that $\Omega\subset\mathbb{R}^{2}$ so we have a 2 dimensional problem. The analysis follows in almost exactly the same way for $\mathbb{R}^{d}$. We will assume that $\Omega$ is a convex polygonal domain (the polygonal assumption can easily be relaxed). The convexity assumption gives us that the $H^2(\Omega)$ norm of the variational solution of our PDE is controlled by the $L^2$ norm of the RHS. We now take our FEM mesh to be a triangulation of $\Omega$ with $h$ being the maximum side length of any triangle in the triangulation. We require a further technical assumption that the meshes we consider remain regular in the sense that as we refine the mesh by decreasing $h$ to zero the angles of all the triangles are bounded below independently of $h$. Since $m_1=\mathcal{L}^{-1}\bar{f}$ is the solution to the following elliptic boundary value problem:
\begin{align*}
    \begin{split}
        -\nabla\cdot(a(x)\nabla v(x))&=\bar{f}(x), \hspace{0.3cm} x\in\Omega \\
        v&=0, \hspace{0.3cm} x\in\partial\Omega
    \end{split}
\end{align*}
and since $m_2=\Phi^{*}M^{-1}\bar{F}=\Phi^{*}M^{-1}\mathcal{I}\bar{f}$ is the FEM solution to the variational formulation of the above problem the error analysis of FEM transfers over to allow us to bound the norm of the difference of the means as follows:
\begin{equation}
    \label{bound_on_diff_means}
    \|m_1-m_2\|_{L^{2}(\Omega)}\leq Ch^{2}\|m_{1}\|_{H^{2}(\Omega)}\leq\tilde{C}h^{2}\|\bar{f}\|_{L^{2}(\Omega)}
\end{equation}
for some constants $C,\tilde{C}>0$. We have utilised in the last inequality above the assumption that the $H^2$ norm of the true solution can be controlled by the $L^2$ norm of $\bar{f}$. The assumptions and error analysis is taken from Chapter 5 of \cite{larsson2008partial} (see in particular Theorem 5.4). \\

We now move on to getting an upperbound for the second term. Using the link with the Procrustes distance discussed above we have:
\begin{equation*}
    W^2(\mathcal{N}(0,\Sigma_1),\mathcal{N}(0,\Sigma_2))\leq\|\Sigma_{1}^{1/2}-\Sigma_{2}^{1/2}\|_{2}^{2}
\end{equation*}

The RHS of the above is still difficult to deal with so we make use of Lemma 4.1 from \cite{powers1970free} to obtain:
\begin{equation*}
    W^2(\mathcal{N}(0,\Sigma_1),\mathcal{N}(0,\Sigma_2))\leq\|\Sigma_{1}^{1/2}-\Sigma_{2}^{1/2}\|_{2}^{2}\leq\|\Sigma_1-\Sigma_2\|_{1}
\end{equation*}
where $\|\cdot\|_1$ is the trace norm or nuclear norm defined by $\|A\|_{1}=\operatorname{tr}(\sqrt{A^{*}A})$.

We now investigate this term:
\begin{align*}
    \|\Sigma_{1}-\Sigma_2\|_{1}&=\|\mathcal{L}^{-1}K(\mathcal{L}^{-1})^{*}-\Phi^{*}M^{-1}K_{\mathcal{I}}M^{-1}\Phi\|_{1} \\
    &=\|\mathcal{L}^{-1}K(\mathcal{L}^{-1})^{*}-\Phi^{*}M^{-1}\mathcal{I}K\mathcal{I}^{*}M^{-1}\Phi\|_{1} \\
    &=\|\mathcal{L}^{-1}K(\mathcal{L}^{-1})^{*}-\Phi^{*}M^{-1}\mathcal{I}K(\Phi^{*}M^{-1}\mathcal{I})^{*}\|_{1}
\end{align*}
where we have used the defintion of $K_{\mathcal{I}}$ and the fact that $M$ and hence $M^{-1}$ is self-adjoint. From this simplification we can see that we can control how ``close" the two covariance operators are because we can control how ``close" $\Phi^{*}M^{-1}\mathcal{I}$ is to $\mathcal{L}^{-1}$. To be more precise:
\begin{align*}
    \|\Sigma_{1}-\Sigma_{2}\|_{1}&=\|\mathcal{L}^{-1}K(\mathcal{L}^{-1})^{*}-\Phi^{*}M^{-1}\mathcal{I}K(\Phi^{*}M^{-1}\mathcal{I})^{*}\|_{1} \\
    &=\|\mathcal{L}^{-1}K(\mathcal{L}^{-1})^{*}-\Phi^{*}M^{-1}\mathcal{I}K(\mathcal{L}^{-1})^{*}+\Phi^{*}M^{-1}\mathcal{I}K(\mathcal{L}^{-1})^{*}-\Phi^{*}M^{-1}\mathcal{I}K(\Phi^{*}M^{-1}\mathcal{I})^{*}\|_{1} \\
    &\leq\|\mathcal{L}^{-1}K(\mathcal{L}^{-1})^{*}-\Phi^{*}M^{-1}\mathcal{I}K(\mathcal{L}^{-1})^{*}\|_{1} + \|\Phi^{*}M^{-1}\mathcal{I}K(\mathcal{L}^{-1})^{*}-\Phi^{*}M^{-1}\mathcal{I}K(\Phi^{*}M^{-1}\mathcal{I})^{*}\|_{1} \\
    &= \|(\mathcal{L}^{-1}-\Phi^{*}M^{-1}\mathcal{I})K(\mathcal{L}^{-1})^{*}\|_{1} + \|\Phi^{*}M^{-1}\mathcal{I}K(\mathcal{L}^{-1}-\Phi^{*}M^{-1}\mathcal{I})^{*}\|_{1} \\
    &\leq\|\mathcal{L}^{-1}-\Phi^{*}M^{-1}\mathcal{I}\|_{\infty}\|K\mathcal{L}^{-1}\|_{1}+\|\Phi^{*}M^{-1}\mathcal{I}K\|_{1}\|(\mathcal{L}^{-1}-\Phi^{*}M^{-1}\mathcal{I})^{*}\|_{\infty} \\
    &\leq\|\mathcal{L}^{-1}-\Phi^{*}M^{-1}\mathcal{I}\|_{\infty}(\|K\mathcal{L}^{-1}\|_{1}+\|\Phi^{*}M^{-1}\mathcal{I}\|_{1}\|K\|_{1})
\end{align*}
where we have utilized Holder's inequality and the sub-multiplicativity of Schatten-p norms (\textit{note: the} $\|\cdot\|_{\infty}$ \textit{is the operator norm here and is a special case of the Schatten-p norm for p being infinity}). As mentioned, we can control how ``close" $\Phi^{*}M^{-1}\mathcal{I}$ is to $\mathcal{L}^{-1}$. To do this we note that (\ref{bound_on_diff_means}) holds for all $\bar{f}$ in $L^{2}(\Omega)$, i.e. we have:
\begin{equation*}
    \|m_1-m_2\|_{L^{2}(\Omega)}=\|(\mathcal{L}^{-1}-\Phi^{*}M^{-1}\mathcal{I})\bar{f}\|_{L^{2}(\Omega)}\leq \tilde{C}h^{2}\|\bar{f}\|_{L^2(\Omega)} \hspace{0.25cm} \forall\bar{f}\in L^{2}(\Omega)
\end{equation*}

This implies that we can bound the operator norm\footnote{Note that here we are viewing both $\mathcal{L}^{-1}$ and $\Phi^{*}M^{-1}\mathcal{I}$ as operators from $L^{2}(\Omega)$ to itself. They are however more generally operators from $H^{-1}(\Omega)$ to $H^{1}_{0}(\Omega)$. We will consider here the case that $\bar{f}$ is in $L^{2}(\Omega)$ but this assumption can be relaxed.} of $\mathcal{L}^{-1}-\Phi^{*}M^{-1}\mathcal{I}$ by $\tilde{C}h^{2}$, i.e.,
\begin{equation}
    \label{bound_on_diff_of_solution_operators}
    \|\mathcal{L}^{-1}-\Phi^{*}M^{-1}\mathcal{I}\|_{\infty}\leq \tilde{C}h^{2}
\end{equation}

Utilising this upper bound together with the fact that $\|\Phi^{*}M^{-1}\mathcal{I}\|_{1}$ is bounded (!!do I need to expand on this!!) we have:
\begin{equation}
    \label{bound_on_diff_cov_operators}
    W^{2}(\mathcal{N}(0,\Sigma_1),\mathcal{N}(0,\Sigma_2))\leq\|\Sigma_{1}-\Sigma_{2}\|_{1}\leq \tilde{C}h^{2}\|K\mathcal{L}^{-1}\|_{1}
\end{equation}

Combining (\ref{bound_on_diff_means}) and (\ref{bound_on_diff_cov_operators}) we now have:
\begin{align*}
    W^{2}(\nu_{1},\nu_{2})&=\|m_1-m_2\|_{L^{2}(\Omega)}^{2}+W^{2}(\mathcal{N}(0,\Sigma_1),\mathcal{N}(0,\Sigma_2)) \\
    &\leq \tilde{C}^{2}h^4\|\bar{f}\|^{2}_{L^{2}(\Omega)} + \tilde{C}h^{2}\|K\mathcal{L}^{-1}\|_{1}
\end{align*}

We thus have an upper bound on the Wasserstein distance between $\nu_{1},\nu_{2}$ in terms of the FEM mesh size $h$:
\begin{equation}
    \label{bound_on_distance}
    W(\nu_{1},\nu_{2})\leq h\sqrt{\tilde{C}\|K\mathcal{L}^{-1}\|_{1}+\tilde{C}^{2}h^{2}\|\bar{f}\|^{2}_{L^{2}(\Omega)}}\leq\gamma h
\end{equation}
where $\gamma>0$ is a constant.
