We now provide details for some of the claims in Notes 1 and 2 above. \\

To start we provide justification for why it is valid to condition as we did in (\ref{conditionalDistnFixed_f}). The justification for this is essentially given by Theorem 3.3 in \cite{owhadi2015conditioning}. To apply this theorem to our problem we note that our pair $(u,\mathcal{I}\mathcal{L}u)^{T}$ lies in the orthogonal direct sum $\mathcal{H}=\mathcal{H}_1\oplus\mathcal{H}_{2}$ of the separable Hilbert spaces $\mathcal{H}_{1}:=H^{1}(\Omega)\times\{0\}$ and $\mathcal{H}_{2}:=\{0\}\times\mathbb{R}^{J}$. The corresponding Gaussian measure we are dealing with on this $\mathcal{H}$ is given by (\ref{jointInfoDist}). Theorem 3.3 states that the conditional distribution is a Gaussian measure with covariance operator being the short of the covariance operator of the distribution (\ref{jointInfoDist}) to $\mathcal{H}_{2}$. This operator,
\begin{equation}
    \begin{pmatrix}
    V & V\mathcal{L}^{*}\mathcal{I}^{*} \\
    \mathcal{I}\mathcal{L}V & \mathcal{I}\mathcal{L}V\mathcal{L}^{*}\mathcal{I}^{*}
    \end{pmatrix}
\end{equation}
is already written in its $(\mathcal{H}_1,\mathcal{H}_2)$ partition representation. In the section on shorted operators in \cite{owhadi2015conditioning} it is stated that provided the lower right partition of the covariance operator of (\ref{jointInfoDist}) corresponding to the covariance in $\mathcal{H}_2$, i.e. $\mathcal{I}\mathcal{L}V(\mathcal{I}\mathcal{L})^{*}$ is invertible then the shorted operator is given by:
\begin{equation}
    \begin{pmatrix}
        V-V\mathcal{L}^{*}\mathcal{I}^{*}(\mathcal{I}\mathcal{L}V\mathcal{L}^{*}\mathcal{I}^{*})^{-1}\mathcal{I}\mathcal{L}V & 0 \\
        0 & 0
    \end{pmatrix}=\begin{pmatrix}
                    \Sigma & 0 \\
                    0 & 0
                  \end{pmatrix}
\end{equation}
We assume the invertiblity of this partition. We do note however that this part of the partition reduces to the $J\times J$ matrix $M\Lambda M^{*}$ when we choose the FEM prior and FEM information operator. The invertibility of this matrix is then equivalent to the invertibility of the Galerkin Matrix, $M$ (if we assume that the entries of $\Lambda$ are non-zero). \\

We thus see that the conditional distribution (\ref{conditionalDistnFixed_f}) has the correct covariance operator. \\

Theorem 3.3 also goes on to give the mean of the conditional distribution in the case that the covariance operator of the joint distribution is compatible with $\mathcal{H}_{2}$. For our problem this is in fact the case as we now show. We first denote the covariance operator of the joint distribution by:
\begin{equation}
    \begin{pmatrix}
        V & V\mathcal{L}^{*}\mathcal{I}^{*} \\
        \mathcal{I}\mathcal{L}V & \mathcal{I}\mathcal{L}V\mathcal{L}^{*}\mathcal{I}^{*}
    \end{pmatrix}=
    \begin{pmatrix}
        C_{11} & C_{12} \\
        C_{21} & C_{22}
    \end{pmatrix}
\end{equation}
i.e. $C_{11}=V, C_{12}=V\mathcal{L}^{*}\mathcal{I}^{*}, C_{21}=\mathcal{I}\mathcal{L}V, C_{22}=\mathcal{I}\mathcal{L}V\mathcal{L}^{*}\mathcal{I}^{*}$. If we now define the following bounded operator $Q:\mathcal{H}\rightarrow\mathcal{H}$ by

\begin{equation}
    Q = \begin{pmatrix}
            0 & 0 \\
            C_{22}^{-1}C_{21} & I
        \end{pmatrix} = \begin{pmatrix}
                            0 & 0 \\
                            \hat{Q} & I
                        \end{pmatrix}
\end{equation}
where $\hat{Q}:=C_{22}^{-1}C_{21}:\mathcal{H}_{1}\rightarrow\mathcal{H}_{2}$ we can easily check that such an operator $Q$ is an (C-symmetric) oblique projection onto $\mathcal{H}_{2}$. Since $Q$ is not $0$ we have by Theorem 3.3 that the mean of the conditional distribution (\ref{conditionalDistnFixed_f}) is given by $\hat{Q}^{*}F$ where the adjoint is defined by the relation $\left\langle\hat{Q}^{*} h_{2}, h_{1}\right\rangle_{\mathcal{H}_{1}}=\left\langle h_{2}, \hat{Q} h_{1}\right\rangle_{\mathcal{H}_{2}}$ for all $h_1\in\mathcal{H}_1,h_{2}\in\mathcal{H}_{2}$. Since we have that $C_{22}$ and hence $C_{22}^{-1}$ is self-adjoint and further that $C_{12}^{*}=C_{21}$ we can easily deduce that $\hat{Q}^{*}=C_{12}C_{22}^{-1}$ and so the mean is given by
\begin{equation}
    \hat{Q}^{*}F=C_{12}C_{22}^{-1}C_{21}F=V\mathcal{L}^{*}\mathcal{I}^{*}(\mathcal{I}\mathcal{L}V\mathcal{L}^{*}\mathcal{I}^{*})^{-1}F=a
\end{equation}
where $a$ is what we had in (\ref{post_mean_before_averaging}). \\

We can now move on to provide conditions for the Galerkin matrix $M$ defined in Notes 2 to be invertible. A sufficient condition to ensure this invertibility is the following assumption (see Assumption 2.34 in \cite{lord2014introduction}): \\

\noindent \textbf{Assumption} (regularity of coefficients) The diffusion coefficient $a(x)$ satisfies:
\begin{equation}
    \label{regularity_of_coeff}
    0<a_{\text{min}}\leq a(x) \leq a_{\text{max}}<\infty \text{ for almost all } x\in\Omega
\end{equation}
for some real constants $a_{\text{min}},a_{\text{max}}$. In particular $a\in L^{\infty}(\Omega)$. \\

If we assume that this assumption holds then it is a simple exercise to show that the following bilinear form from $H_{0}^{1}(\Omega)\times H_{0}^{1}(\Omega)$ to $\mathbb{R}$ defined by:
\begin{equation}
    \label{bilinear_form}
    \tilde{a}(u,w):=\int_{\Omega}a(x)\nabla u(x)\cdot\nabla w(x)\mathrm{d}x
\end{equation}
defines a norm $|\cdot|_{E}$ on $H_{0}^{1}(\Omega)$ via $|u|_{E}:=\tilde{a}(u,u)^{1/2}$. To show that the Galerkin matrix $M$ is invertible we will show that it is (strictly) positive definite. To this end let $\mathbf{v}\in\mathbb{R}^{J}\backslash\{0\}$ and define $v=\sum_{i=1}^{J}v_{i}\psi_{i}$ which is a function in $H_{0}^{1}(\Omega)$. We now have:
\begin{align*}
    \mathbf{v}^{T}(M)\mathbf{v}&=\sum_{i,j=1}^{J}v_{j}(M_{ij})v_{i} \\
    &=\sum_{i,j=1}^{J}v_{j}\left(\int_{\Omega}a(x)\nabla\psi_{i}(x)\cdot\nabla\psi_{j}(x)\mathrm{d}x\right)v_{i} \\
    &=\sum_{i,j=1}^{J}v_{j}\tilde{a}(\psi_{i},\psi_{j})v_{i}\\
    &=\tilde{a}\left(\sum_{i=1}^{J}v_{i}\psi_{i},\sum_{j=1}^{J}v_{j}\psi_{j}\right) \\
    &=\tilde{a}(v,v) \\
    &=|v|_{E}^{2}>0
\end{align*}
where the strict inequality follows since $\tilde{a}(v,v)=0$ iff $|v|_{E}=0$ iff $v=0$ (since $|\cdot|_{E}$ is a norm on $H_{0}^{1}(\Omega)$) iff $\mathbf{v}=0$ since the $\{\psi_{i}\}$ are linearly independent.
