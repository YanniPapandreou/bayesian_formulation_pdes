In this section we provide the omitted details for the proofs and methodology presented in Sections \textcolor{blue}{\ref{general_framework}} and \textcolor{blue}{\ref{elliptic_bv_prob}}.

\subsection{Justification of the conditioning step yielding equation (\ref{conditionalDistnFixed_f}) in Section \textcolor{blue}{\ref{general_framework}}}

We now provide a full justification for the conditioning step yielding equation (\ref{conditionalDistnFixed_f}) in Section \textcolor{blue}{\ref{general_framework}}. The justification essentially follows from Theorem 3.3 in \textcolor{blue}{\cite{owhadi2015conditioning}}. This theorem is given below (we only give the part of the theorem we require): \vspace{10pt}

\begin{theorem}
    \label{condition_gaussian_hilbert}
    Consider a Gaussian measure $\mu$ on an orthogonal direct sum $H=H_{1}\oplus H_{2}$ of separable Hilbert spaces with mean $m$ and covariance operator $C$. Then for all $t\in H_2$, the conditional measure $\mu_{t}$ is a Gaussian measure with covariance operator $\mathcal{H}_{1}(C)$, the short of $C$ to $H_{1}$.

    If the covariance operator $C$ is compatible with $H_2$, then for any oblique projection $Q$ in $\mathcal{P}(C,H_2)\neq\emptyset$, the mean $m_t$ of the conditional measure $\mu_t$ is
    \begin{equation*}
        m_t = \begin{pmatrix}
                m_1 + \hat{Q}^{*}(t-m_2) \\
                t
              \end{pmatrix}
    \end{equation*}
    Where \begin{equation*}
            Q=\begin{pmatrix}
                0 & 0 \\
                \hat{Q} & I
              \end{pmatrix}
          \end{equation*}
   and $\mathcal{P}(C,H_2)$ is the set of (C-symmetric) oblique projections onto $H_2$. \textit{Note: the terms shorted operator, compatible and oblique projections are all defined in} \textcolor{blue}{\cite{owhadi2015conditioning}}.
\end{theorem}
To apply this theorem to our problem we note that our pair $(u,\mathcal{I}\mathcal{L}u)^{T}$ lies in the orthogonal direct sum $\mathcal{H}_{1}\oplus\mathcal{H}_{2}$ of the separable Hilbert spaces $\mathcal{H}_{1}:=\mathcal{H}\times\{0\}$ and $\mathcal{H}_{2}:=\{0\}\times\mathbb{R}^{J}$. The corresponding Gaussian measure we are dealing with on this space is given by (\ref{jointInfoDist}). Theorem \textcolor{blue}{\ref{condition_gaussian_hilbert}} states that the conditional distribution is a Gaussian measure with covariance operator being the short of the covariance operator of the distribution (\ref{jointInfoDist}) to $\mathcal{H}_{2}$. This operator,
\begin{equation*}
    \begin{pmatrix}
        V & V\mathcal{L}^{*}\mathcal{I}^{*} \\
        \mathcal{I}\mathcal{L}V & \mathcal{I}\mathcal{L}V\mathcal{L}^{*}\mathcal{I}^{*}
    \end{pmatrix}
\end{equation*}
is already written in its $(\mathcal{H}_{1},\mathcal{H}_{2})$ partition representation. In the section on shorted operators in \textcolor{blue}{\cite{owhadi2015conditioning}} it is stated that provided the lower right partition of the covariance operator of (\ref{jointInfoDist}) corresponding to the covariance in $\mathcal{H}_{2}$, i.e. $\mathcal{I}\mathcal{L}V(\mathcal{I}\mathcal{L})^{*}$, is invertible then the shorted operator is given by:
\begin{equation*}
    \begin{pmatrix}
        V-V\mathcal{L}^{*}\mathcal{I}^{*}(\mathcal{I}\mathcal{L}V\mathcal{L}^{*}\mathcal{I}^{*})^{-1}\mathcal{I}\mathcal{L}V & 0 \\
        0 & 0
    \end{pmatrix}
\end{equation*}
We assume the invertibility of this partition. \vspace{10pt}

\begin{remark}
    This part of the partition reduces to the $J\times J$ matrix $A\Lambda A^{*}$ when we choose the FEM prior and FEM information operator as was discussed in Section \textcolor{blue}{\ref{elliptic_bv_prob}}. The invertibility of this matrix is then equivalent to the invertibility of the Galerkin Stiffness Matrix $A$ (since we assume the entries of $\Lambda$ are non-zero). $\mathbin{\blacklozenge}$
\end{remark}
We thus see that the conditional distribution (\ref{conditionalDistnFixed_f}) has the correct covariance operator. Theorem \textcolor{blue}{\ref{condition_gaussian_hilbert}} also goes on to give the mean of the conditional distribution in the case that the covariance operator of the joint distribution is compatible with $H_{2}$. For our problem this is in fact the case as we now show. We first denote the covariance operator of the joint distribution by:
\begin{equation*}
    \begin{pmatrix}
        V & V\mathcal{L}^{*}\mathcal{I}^{*} \\
        \mathcal{I}\mathcal{L}V & \mathcal{I}\mathcal{L}V\mathcal{L}^{*}\mathcal{I}^{*}
    \end{pmatrix}=
    \begin{pmatrix}
        C_{11} & C_{12} \\
        C_{21} & C_{22}
    \end{pmatrix}
\end{equation*}
i.e. $C_{11}:=V, C_{12}:=V\mathcal{L}^{*}\mathcal{I}^{*},C_{21}=\mathcal{I}\mathcal{L}V,C_{22}:=\mathcal{I}\mathcal{L}V\mathcal{L}^{*}\mathcal{I}^{*}$. If we now define the following bounded operator $Q:H\rightarrow H$ by
\begin{equation*}
    Q = \begin{pmatrix}
            0 & 0 \\
            C_{22}^{-1}C_{21} & I
        \end{pmatrix} =
        \begin{pmatrix}
            0 & 0 \\
            \hat{Q} & I
        \end{pmatrix}
\end{equation*}
where $\hat{Q}:=C_{22}^{-1}C_{21}:H_{1}\rightarrow H_{2}$ we can easily check that such an operator $Q$ is an (C-symmetric) oblique projection onto $H_{2}$. Since $Q$ is not $0$ we have by Theorem \textcolor{blue}{\ref{condition_gaussian_hilbert}} that the mean of the conditional distribution (\ref{conditionalDistnFixed_f}) is given by $\hat{Q}^{*}F$. Now $C_{22}$ and therefore $C_{22}^{-1}$ are self-adjoint and further we have $C_{12}^{*}=C_{21}$ and so we can easily deduce that $\hat{Q}^{*}=C_{12}C_{22}^{-1}$ and so the mean is given by:
\begin{equation}
    \hat{Q}^{*}F=C_{12}C_{22}^{-1}F=V\mathcal{L}^{*}\mathcal{I}^{*}(\mathcal{I}\mathcal{L}V\mathcal{L}^{*}\mathcal{I}^{*})^{-1}F=a
\end{equation}
where $a$ is what we had in (\ref{post_mean_before_averaging}). We thus see that the conditioning step is justified and we indeed obtain what was claimed in Section \textcolor{blue}{\ref{general_framework}}. \qedsymbol

\subsection{Detailed computations for the proof of Proposition \textcolor{blue}{\ref{first_prop}}}

We provide here details of the computation of the following integral needed for the proof of Proposition \textcolor{blue}{\ref{first_prop}}:
\begin{equation*}
    \int\int\psi(u^{N})\mu_{PQF,\Sigma_N}(\mathrm{d}u^{N})\mu_{\bar{F},K_{\mathcal{I}}}(\mathrm{d}F)
\end{equation*}
Both measures in the above integral are multivariate normal and so we have:
\begin{align}
    \int\int&\psi(u^{N})\mu_{Pa,\Sigma_N}(\mathrm{d}u^{N})\mu_{\bar{F},K_{\mathcal{I}}}(\mathrm{d}F)= \nonumber \\ &=\frac{1}{Z_u}\int\int\psi(u^{N})\exp\left(-\frac{1}{2}\left\langle u^{N}-PQF,\Sigma_{N}^{-1}(u^{N}-PQF) \right\rangle\right)\mu_{\bar{F},K_{\mathcal{I}}}(\mathrm{d}F)\mathrm{d}u^{N} \nonumber \\
    &=\frac{1}{Z_{u}Z_{f}}\int\int\psi(u^{N})\exp\left(-\frac{1}{2}\left\langle u^{N}-PQF,\Sigma_{N}^{-1}(u^{N}-PQF)\right\rangle\right)\exp\left(-\frac{1}{2}\left\langle F-\bar{F},K_{\mathcal{I}}^{-1}(F-\bar{F}) \right\rangle\right)\mathrm{d}F\mathrm{d}u^{N}
\end{align}
where the normalization constants are $Z_u:=(2\pi)^{N/2}\det(\Sigma_N)^{1/2}$ and $Z_f:=(2\pi)^{J/2}\det(K_\mathcal{I})^{1/2}$. We now need to compute the integral over $F$. In order to do so we combine the exponents in (11) into a quadratic in $F$ in order to be able to use the well-known formula of a multidimensional Gaussian integral. Going through the algebra we obtain:

\begin{align}
    \frac{1}{Z_{u}Z_{f}}&\int\int\psi(u^{N})\exp\Bigg(-\frac{1}{2}\Big(\left\langle u^{N},\Sigma_{N}^{-1}u^{N} \right\rangle - 2\left\langle \Sigma_{N}^{-1}PQF,u^{N}\right\rangle + \left\langle F,Q^{*}P^{*}\Sigma_{N}^{-1}PQF\right\rangle \nonumber \\
    &+ \left\langle F,K_{\mathcal{I}}^{-1}F\right\rangle - 2 \left\langle K_{\mathcal{I}}^{-1}\bar{F},F \right\rangle + \left\langle \bar{F},K_{\mathcal{I}}^{-1}\bar{F}\right\rangle \Big)\Bigg)\mathrm{d}F\mathrm{d}u^{N} = \nonumber \\
    &=\frac{1}{Z_{u}Z_{f}}\int\psi(u^{N})\exp\left(-\frac{1}{2}\left(\left\langle u^{N},\Sigma_{N}^{-1} u^{N} \right\rangle + \left\langle \bar{F},K_{\mathcal{I}}^{-1}\bar{F} \right\rangle\right)\right) \cdot \nonumber \\
    &\left(\int\exp\left(-\frac{1}{2}\left\langle F,BF \right\rangle + \left\langle Q^{*}P^{*}\Sigma_{N}^{-1}u^{N} + K_{\mathcal{I}}^{-1}\bar{F},F \right\rangle\right)\mathrm{d}F\right)\mathrm{d}u^{N}
\end{align}
where $B:=\left(Q^{*}P^{*}\Sigma_{N}^{-1}PQ+K_{\mathcal{I}}^{-1}\right)$. Computing the inner integral over $F$ we thus obtain:
\begin{align}
    \frac{(2\pi)^{J/2}}{Z_{u}Z_{f}\det(B)^{1/2}}\int\psi(u^{N})\exp&\left(\frac{1}{2}\left\langle Q^{*}P^{*}\Sigma_{N}^{-1}u^{N} + K_{\mathcal{I}}^{-1}\bar{F}, B^{-1}\left(Q^{*}P^{*}\Sigma_{N}^{-1}u^{N} + K_{\mathcal{I}}^{-1}\bar{F}\right) \right\rangle\right. \nonumber \\
    &\left.-\frac{1}{2}\left(\left\langle u^{N}, \Sigma_{N}^{-1}u^{N}\right\rangle + \left\langle \bar{F},K_{\mathcal{I}}^{-1}\bar{F} \right\rangle \right)\right)\mathrm{d}u^{N}
\end{align}
We now focus on the terms in the exponent and simplify these as follows:
\begin{align}
    &\frac{1}{2}\left\langle Q^{*}P^{*}\Sigma_{N}^{-1}u^{N} + K_{\mathcal{I}}^{-1}\bar{F}, B^{-1}\left(Q^{*}P^{*}\Sigma_{N}^{-1}u^{N} + K_{\mathcal{I}}^{-1}\bar{F}\right) \right\rangle -\frac{1}{2}\left(\left\langle u^{N}, \Sigma_{N}^{-1}u^{N}\right\rangle + \left\langle \bar{F},K_{\mathcal{I}}^{-1}\bar{F} \right\rangle \right) =  \nonumber \\
    &=-\frac{1}{2}\Big( \left\langle u^{N}, \Sigma_{N}^{-1}u^{N}\right\rangle + \left\langle \bar{F}, K_{\mathcal{I}}^{-1}\bar{F} \right\rangle - \left\langle Q^{*}P^{*}\Sigma_{N}^{-1}u^{N} + K_{\mathcal{I}}^{-1}\bar{F}, B^{-1}\left(Q^{*}P^{*}\Sigma_{N}^{-1}u^{N} + K_{\mathcal{I}}^{-1}\bar{F}\right) \right\rangle \Big) \nonumber \\
    &=-\frac{1}{2}\Big(\left\langle u^{N}, \Sigma_{N}^{-1}u^{N}\right\rangle + \left\langle \bar{F}, K_{\mathcal{I}}^{-1}\bar{F} \right\rangle - \left\langle u^{N},\Sigma_{N}^{-1}PQB^{-1}Q^{*}P^{*}\Sigma_{N}^{-1}u^{N} \right\rangle \nonumber \\
    &- 2\left\langle u^{N},\Sigma_{N}^{-1}PQB^{-1}K_{\mathcal{I}}^{-1}\bar{F} \right\rangle - \left\langle \bar{F}, K_{\mathcal{I}}^{-1}B^{-1}K_{\mathcal{I}}^{-1}\bar{F} \right\rangle \Big) \nonumber \\
    &= -\frac{1}{2}\Big( \left\langle u^{N}, (\Sigma_{N}^{-1} - \Sigma_{N}^{-1}PQB^{-1}Q^{*}P^{*}\Sigma_{N}^{-1})u^{N} \right\rangle - 2 \left\langle u^{N},\Sigma_{N}^{-1}PQB^{-1}K_{\mathcal{I}}^{-1}\bar{F} \right\rangle + \left\langle \bar{F}, (K_{\mathcal{I}}^{-1}-K_{\mathcal{I}}^{-1}B^{-1}K_{\mathcal{I}}^{-1})\bar{F} \right\rangle \Big) \nonumber \\
    &=-\frac{1}{2}\Big( \left\langle u^{N}, \Sigma_{\mathcal{I}}^{-1}u^{N} \right\rangle - 2 \left\langle u^{N},\Sigma_{N}^{-1}PQB^{-1}K_{\mathcal{I}}^{-1}\bar{F} \right\rangle + \left\langle \bar{F}, (K_{\mathcal{I}}^{-1}-K_{\mathcal{I}}^{-1}B^{-1}K_{\mathcal{I}}^{-1})\bar{F} \right\rangle \Big) \label{exponent}
\end{align}
where $\Sigma_{\mathcal{I}}:=\Sigma_{N} + PQK_{\mathcal{I}}Q^{*}P^{*}$. The inverse of $\Sigma_{\mathcal{I}}$ is indeed the coefficient matrix for the quadratic term in $u^{N}$ in (\ref{exponent}). This can be seen by utilizing the Woodbury matrix identity as follows:
\begin{align*}
    \Sigma_{\mathcal{I}}^{-1}&=(\Sigma_{N} + PQK_{\mathcal{I}}Q^{*}P^{*})^{-1} \\
    &=\Sigma_{N}^{-1}-\Sigma_{N}^{-1}PQ(K_{\mathcal{I}}^{-1}+Q^{*}P^{*}\Sigma_{N}^{-1}PQ)^{-1}Q^{*}P^{*}\Sigma_{N}^{-1} \\
    &=\Sigma_{N}^{-1}-\Sigma_{N}^{-1}PQB^{-1}Q^{*}P^{*}\Sigma_{N}^{-1}
\end{align*}
We now complete the square (in terms of $u^{N}$) to obtain:
\begin{align}
    -&\frac{1}{2}\Big(\left\langle u^{N}-h^{N},\Sigma_{\mathcal{I}}^{-1}(u^{N}-h^{N}) \right\rangle + \left\langle \bar{F}, (K_{\mathcal{I}}^{-1}-K_{\mathcal{I}}^{-1}B^{-1}K_{\mathcal{I}}^{-1})\bar{F} \right\rangle - \left\langle \Sigma_{N}^{-1}PQB^{-1}K_{\mathcal{I}}^{-1}\bar{F},\Sigma_{\mathcal{I}}\Sigma_{N}^{-1}PQB^{-1}K_{\mathcal{I}}^{-1}\bar{F} \right\rangle \Big) = \nonumber \\
    &=-\frac{1}{2}\Big(\left\langle u^{N}-h^{N},\Sigma_{\mathcal{I}}^{-1}(u^{N}-h^{N}) \right\rangle + \left\langle \bar{F}, (K_{\mathcal{I}}^{-1}-K_{\mathcal{I}}^{-1}B^{-1}K_{\mathcal{I}}^{-1}-K_{\mathcal{I}}^{-1}B^{-1}Q^{*}P^{*}\Sigma_{N}^{-1}\Sigma_{\mathcal{I}}\Sigma_{N}^{-1}PQB^{-1}K_{\mathcal{I}}^{-1})\bar{F} \right\rangle\Big)
\end{align}
where $h^{N}:=\Sigma_{\mathcal{I}}\Sigma_{N}^{-1}PQB^{-1}K_{\mathcal{I}}^{-1}\bar{F}$. We now show that the term quadratic in $\bar{F}$ vanishes by showing that the coefficient matrix of $\bar{F}$ is equal to the zero matrix. Note that this coefficent matrix can be rewritten as:
\begin{align*}
    &K_{\mathcal{I}}^{-1}-K_{\mathcal{I}}^{-1}B^{-1}K_{\mathcal{I}}^{-1}-K_{\mathcal{I}}^{-1}B^{-1}Q^{*}P^{*}\Sigma_{N}^{-1}\Sigma_{\mathcal{I}}\Sigma_{N}^{-1}PQB^{-1}K_{\mathcal{I}}^{-1} = \\
    &=K_{\mathcal{I}}^{-1}K_{\mathcal{I}}K_{\mathcal{I}}^{-1}-K_{\mathcal{I}}^{-1}B^{-1}K_{\mathcal{I}}^{-1}-K_{\mathcal{I}}^{-1}B^{-1}Q^{*}P^{*}\Sigma_{N}^{-1}\Sigma_{\mathcal{I}}\Sigma_{N}^{-1}PQB^{-1}K_{\mathcal{I}}^{-1} \\
    &= K_{\mathcal{I}}^{-1}(K_{\mathcal{I}}-B^{-1}-B^{-1}Q^{*}P^{*}\Sigma_{N}^{-1}\Sigma_{\mathcal{I}}\Sigma_{N}^{-1}PQB^{-1})K_{\mathcal{I}}^{-1} \\
    &=K_{\mathcal{I}}^{-1}B^{-1}(BK_{\mathcal{I}}B-B-Q^{*}P^{*}\Sigma_{N}^{-1}\Sigma_{\mathcal{I}}\Sigma_{N}^{-1}PQ)B^{-1}K_{\mathcal{I}}^{-1}
\end{align*}
and so showing that it is the zero matrix is equivalent to showing that $(BK_{\mathcal{I}}B-B-Q^{*}P^{*}\Sigma_{N}^{-1}\Sigma_{\mathcal{I}}\Sigma_{N}^{-1}PQ)$ is the zero matrix. This can be shown as follows:
\begin{align*}
    &BK_{\mathcal{I}}B-B-Q^{*}P^{*}\Sigma_{N}^{-1}\Sigma_{\mathcal{I}}\Sigma_{N}^{-1}PQ = \\
    &=(Q^{*}P^{*}\Sigma_{N}^{-1}PQ+K_{\mathcal{I}}^{-1})K_{\mathcal{I}}B-B-Q^{*}P^{*}\Sigma_{N}^{-1}\Sigma_{\mathcal{I}}\Sigma_{N}^{-1}PQ \\
    &=Q^{*}P^{*}\Sigma_{N}^{-1}PQK_{\mathcal{I}}B+B-B-Q^{*}P^{*}\Sigma_{N}^{-1}\Sigma_{\mathcal{I}}\Sigma_{N}^{-1}PQ \\
    &=Q^{*}P^{*}\Sigma_{N}^{-1}PQK_{\mathcal{I}}(Q^{*}P^{*}\Sigma_{N}^{-1}PQ+K_{\mathcal{I}}^{-1})-Q^{*}P^{*}\Sigma_{N}^{-1}\Sigma_{\mathcal{I}}\Sigma_{N}^{-1}PQ \\
    &=Q^{*}P^{*}\Sigma_{N}^{-1}PQK_{\mathcal{I}}Q^{*}P^{*}\Sigma_{N}^{-1}PQ + Q^{*}P^{*}\Sigma_{N}^{-1}PQ - Q^{*}P^{*}\Sigma_{N}^{-1}\Sigma_{\mathcal{I}}\Sigma_{N}^{-1}PQ \\
    &= Q^{*}P^{*}\Sigma_{N}^{-1}(PQK_{\mathcal{I}}Q^{*}P^{*}+\Sigma_{N}-\Sigma_{\mathcal{I}})\Sigma_{N}^{-1}PQ = 0
\end{align*}
where the last equality follows by the definition of $\Sigma_{\mathcal{I}}$. Thus, our integral simplifies to:
\begin{equation}
    \frac{(2\pi)^{J/2}}{Z_{u}Z_{f}\det(B)^{1/2}}\int\psi(u^{N})\exp\Big(-\frac{1}{2}\left\langle u^{N}-h^{N},\Sigma_{\mathcal{I}}^{-1}(u^{N}-h^{N}) \right\rangle\Big)\mathrm{d}u^{N}
\end{equation}
We now focus on simplifying the normalizing constants in front of the integral:
\begin{align*}
    \frac{(2\pi)^{J/2}}{Z_{u}Z_{f}\det(B)^{1/2}}&=\frac{(2\pi)^{J/2}}{(2\pi)^{N/2}\det(\Sigma_N)^{1/2}(2\pi)^{J/2}\det(K_\mathcal{I})^{1/2}\det(B)^{1/2}}= \\
    &=\frac{1}{(2\pi)^{N/2}\det(\Sigma_N)^{1/2}\det(K_\mathcal{I})^{1/2}\det(B)^{1/2}}
\end{align*}
To proceed we note that $\det{B}$ can be rewritten as follows:
\begin{align*}
    \det(B)&=\det(Q^{*}P^{*}\Sigma_{N}^{-1}PQ + K_{\mathcal{I}}^{-1}) \\
    &=\det(K_{\mathcal{I}}^{-1}(I+K_{\mathcal{I}}Q^{*}P^{*}\Sigma_{N}^{-1}PQ)) \\
    &=\det(K_{\mathcal{I}}^{-1})\det(I+(K_{\mathcal{I}}Q^{*}P^{*})(\Sigma_{N}^{-1}PQ))) \\
    &=\det(K_{\mathcal{I}})^{-1}\det(I+\Sigma_{N}^{-1}PQK_{\mathcal{I}}Q^{*}P^{*})
\end{align*}
where we have utilized Sylvester's determinant theorem (\textit{note: the identity matrices in the last two lines are of different sizes}). We can now finish up the simplification of the constants outside the integral:
\begin{align*}
    &\frac{1}{(2\pi)^{N/2}\det(\Sigma_N)^{1/2}\det(K_\mathcal{I})^{1/2}\det(B)^{1/2}} \\
    &=\frac{1}{(2\pi)^{N/2}\det(\Sigma_N)^{1/2}\det(K_\mathcal{I})^{1/2}\det(K_{\mathcal{I}})^{-1/2}\det(I+\Sigma_{N}^{-1}PQK_{\mathcal{I}}Q^{*}P^{*})^{1/2}} \\
    &= \frac{1}{(2\pi)^{N/2}\det(\Sigma_{N} + PQK_{\mathcal{I}}Q^{*}P^{*})^{1/2}} \\
    &=\frac{1}{(2\pi)^{N/2}\det(\Sigma_{\mathcal{I}})^{1/2}}
\end{align*}
Thus, our integral becomes:
\begin{align}
    \int&\psi(u^{N})\frac{1}{(2\pi)^{N/2}\det(\Sigma_{\mathcal{I}})^{1/2}}\exp\left(-\frac{1}{2}\left\langle u^{N}-h^{N}, \Sigma_{\mathcal{I}}^{-1}(u^{N}-h^{N}) \right\rangle\right)\mathrm{d}u^N = \nonumber \\
    =&\int\psi(u^{N})\mu_{h^{N},\Sigma_{\mathcal{I}}}(\mathrm{d}u^N)
\end{align}
from which we see that we have obtained the expectation of $\psi$ w.r.t. a multivariate Gaussian with mean and covariance given by:
\begin{equation}
    h^{N}:=\Sigma_{\mathcal{I}}\Sigma_{N}^{-1}PQB^{-1}K_{\mathcal{I}}^{-1}\bar{F}=PQ\bar{F}
\end{equation}
\begin{equation}
    \Sigma_{\mathcal{I}}=P(\Sigma+QK_{\mathcal{I}}Q^{*})P^{*}
\end{equation}

\noindent Note that we have simplified the mean $h^N$ of this multivariate Gaussian as follows:
\begin{align*}
    h^N &= \Sigma_{\mathcal{I}}\Sigma_{N}^{-1}PQB^{-1}K_{\mathcal{I}}^{-1}\bar{F} \\
    &=(\Sigma_N+PQK_{\mathcal{I}}Q^{*}P^{*})\Sigma_{N}^{-1}PQ(K_{\mathcal{I}}B)^{-1}\bar{F} \\
    &=PQ(K_{\mathcal{I}}B)^{-1}\bar{F}+PQK_{\mathcal{I}}Q^{*}P^{*}\Sigma_{N}^{-1}PQ(K_{\mathcal{I}}B)^{-1}\bar{F} \\
    &=PQ(I+K_{\mathcal{I}}Q^{*}P^{*}\Sigma_{N}^{-1}PQ)(K_{\mathcal{I}}B)^{-1}\bar{F} \\
    &=PQ(I+K_{\mathcal{I}}Q^{*}P^{*}\Sigma_{N}^{-1}PQ)(K_{\mathcal{I}}(K_{\mathcal{I}}^{-1}+Q^{*}P^{*}\Sigma_{N}^{-1}PQ))^{-1}\bar{F}\\
    &=PQ(I+K_{\mathcal{I}}Q^{*}P^{*}\Sigma_{N}^{-1}PQ)(I+K_{\mathcal{I}}Q^{*}P^{*}\Sigma_{N}^{-1}PQ)^{-1}\bar{F}\\
    &=PQ\bar{F}
\end{align*}
We thus obtain (\ref{psi_integral_one}) as claimed in the proof of Proposition \textcolor{blue}{\ref{first_prop}}. \qedsymbol

\subsection{Proof of the invertibility of the Galerkin Stiffness matrix $A$ introduced in Section \textcolor{blue}{\ref{elliptic_bv_prob}}}

We prove here that that Galerkin Stiffness matrix $A$ with $ij$-\textit{th} entry given by $A_{ij}:=\int_{\Omega}a(x)\nabla\phi_{i}(x)\cdot\nabla\phi_{j}(x)\mathrm{d}x$ is invertible. The proof comes from \textcolor{blue}{\cite{lord2014introduction}}. As mentioned we assume that the diffusion coefficent $a(x)$ satisfies Assumption 2.

\noindent \textit{Proof:} In order to show that $A$ is invertible we shall show that it is (strictly) positive definite. Under Assumption 2 it is a simple exercise to show that the following bilinear form from $H_{0}^{1}(\Omega)\times H_{0}^{1}(\Omega)$ to $\mathbb{R}$ defined by:
\begin{equation}
    \tilde{a}(u,w):=\int_{\Omega}a(x)\nabla u(x)\cdot\nabla w(x)\mathrm{d}x
\end{equation}
defines a norm $|\boldsymbol{\cdot}|_{E}$ on $H_{0}^{1}(\Omega)$ via $|v|_{E}:=\tilde{a}(u,u)^{1/2}$. We can now show that the Galerkin stiffness matrix $A$ is (strictly) positive definite. To this end let $\mathbf{v}\in\mathbb{R}^{J}\backslash\{0\}$ and define $v=\Phi^{*}\mathbf{v}=\sum_{i=1}^{J}v_{i}\phi_{i}$ where $\Phi^{*}$ is the operator in Section \textcolor{blue}{\ref{elliptic_bv_prob}} defined by equation (\ref{projection_into_fem_space}). Note that $v$ so defined is a function in $H_{0}^{1}(\Omega)$. We can now compute:
\begin{align*}
    \mathbf{v}^{T}A\mathbf{v} &= \sum_{i,j=1}^{J}v_{j}A_{ij}v_{i} \\
    &=\sum_{i,j=1}^{J}v_{j}(\int_{\Omega}a(x)\nabla\phi_{i}(x)\cdot\nabla\phi_{j}(x)\mathrm{d}x)v_{i} \\
    &=\sum_{i,j=1}^{J}v_{j}\tilde{a}(\phi_{i},\phi_{j})v_{i} \\
    &=\tilde{a}\left(\sum_{i=1}^{J}v_{i}\phi_{i},\sum_{j=1}^{J}v_{j}\phi_{j}\right) \\
    &= \tilde{a}(v,v) \\
    &= |v|_{E}^{2}>0
\end{align*}
where the strict inequality follows since $\tilde{a}(v,v)=0$ iff $|v|_{E}=0$ iff $v=0$ (since $|\boldsymbol{\cdot}|_{E}$ is a norm of $H_{0}^{1}(\Omega)$) iff $\mathbf{v}=0$ since the $\{\phi_{i}\}$ are linearly independent. \qedsymbol

\subsection{Detailed computations for proving equation (\ref{approximate_fem_prior}) in Section \textcolor{blue}{\ref{elliptic_bv_prob}}}

We provide here the detailed computation for showing that the averaged distribution is indeed given by (\ref{approximate_fem_prior}). As explained in Section \textcolor{blue}{\ref{elliptic_bv_prob}} we must redo the calculation of the expectation of an arbitrary bounded cylindrical test function. We are thus again interested in computing the following integral:
\begin{equation}
    \int\int\psi(u^{N})\mu_{a,\Sigma}(\mathrm{d}u)\mu_{\bar{f},K}(\mathrm{d}f)
\end{equation}
where now $a=\Phi^{*}A^{-1}F$ and $\Sigma=0$. Thus, $\mu_{a,\Sigma}=\delta_{a}$, that is to say that the measure is in fact a Dirac point mass at $a=\Phi^{*}A^{-1}F$. We thus have:
\begin{equation*}
    \int\int\psi(u^{N})\mu_{a,\Sigma}(\mathrm{d}u)\mu_{\bar{f},K}(\mathrm{d}f)=\int\int\psi(u^N)\delta_{\Phi^{*}A^{-1}F}(\mathrm{d}u)\mu_{\bar{f},K}(\mathrm{d}f)
\end{equation*}
Now since the posterior for $u$ for a fixed realisation of $f$ is a point mass we have that the posterior for $u^N=Pu$ is $\delta_{P\Phi^{*}A^{-1}F}$. Thus, we can write:
\begin{equation*}
    \int\int\psi(u^N)\delta_{\Phi^{*}A^{-1}F}(\mathrm{d}u)\mu_{\bar{f},K}(\mathrm{d}f)=\int\int\phi(u^N)\delta_{P\Phi^{*}A^{-1}F}(\mathrm{d}u^N)\mu_{\bar{f},K}(\mathrm{d}f)
\end{equation*}
It is now important to point out that $P\Phi^{*}$ is to be interpreted\footnote{By this we mean that $P\Phi^{*}$ is actually interpreted as $P\otimes\Phi^{*}$} as the following $N\times J$ matrix:
\begin{align*}
    P\Phi^{*}&=P(\phi_{1},\dots,\phi_{J}) \\
    &=(P\phi_{1},\dots,P\phi_{J}) \\
    &=\begin{pmatrix}
        \phi_{1}(x_1) & \dots & \phi_{J}(x_1) \\
        \vdots & & \vdots \\
        \phi_{1}(x_N) & \dots & \phi_{J}(x_N)
      \end{pmatrix}=:\Theta
\end{align*}
i.e. the matrix with $ij$\textit{th} entry $\Theta_{i,j}:=\phi_{j}(x_{i})$ for $i\in\{1,\dots,N\},j\in\{1,\dots,J\}$. We thus have that $Y:=P\Phi^{*}A^{-1}F=\Theta A^{-1}F\sim\mathcal{N}(\Theta A^{-1}\bar{F},\Theta A^{-1}K_{\mathcal{I}}A^{-1}\Theta^{*})$. Together with this and with the fact that the posterior of $u^{N}$ only depends on $f$ through $F$ we can write:
\begin{align*}
    \int\int\psi(u^N)\delta_{P\Phi^{*}A^{-1}F}(\mathrm{d}u^N)\mu_{\bar{f},K}(\mathrm{d}f)&=\int\int\phi(u^N)\delta_{\Theta A^{-1}F}(\mathrm{d}u^{N})\mu_{\bar{F},K_{\mathcal{I}}}(\mathrm{d}F) \\
    &=\int\int\psi(u^N)\delta_{Y}(\mathrm{d}u^N)\mu_{\Theta A^{-1}\bar{F},\Theta A^{-1}K_{\mathcal{I}}A^{-1}\Theta^{*}}(\mathrm{d}Y) \\
    &=\int\psi(u^N)\left(\int\delta_{Y}(\mathrm{d}u^N)\mu_{\Theta A^{-1}\bar{F},\Theta A^{-1}K_{\mathcal{I}}A^{-1}\Theta^{*}}(\mathrm{d}Y)\right) \\
    &=\int\psi(u^N)\mu_{\Theta A^{-1}\bar{F},\Theta A^{-1}K_{\mathcal{I}}A^{-1}\Theta^{*}}(\mathrm{d}u^N)
\end{align*}
which we recognize as a finite dimensional projection of a Gaussian measure. Thus, we conclude that ``averaging" over $f$ gives the expected Gaussian posterior under these choices:
\begin{equation}
    \label{average_posterior_FEM_prior}
    \mathcal{N}(\Phi^{*}A^{-1}\bar{F},\Phi^{*}A^{-1}K_{\mathcal{I}}A^{-1}\Phi)
\end{equation}
\qedsymbol

\subsection{Detailed computations for the proof of Theorem \textcolor{blue}{\ref{posterior_difference_bound}}}

Detailed computations to obtain the upper-bounds in Theorem \textcolor{blue}{\ref{posterior_difference_bound}} are now presented. For convenience we will denote $\beta:=\|B_{1}^{-1}\|_{\infty}$ which, as mentioned in Section \textcolor{blue}{\ref{elliptic_bv_prob}} is a bounded constant independent of $h$. We will now give a detailed derivation of the upper-bound on the two remaining terms in the bound on $\|m^{(1)}_{u|\mathbf{v}}-m^{(2)}_{u|\mathbf{v}}\|_{L^{2}(\Omega)}$. We start with the second term:
\begin{equation*}
    \|(\Sigma_{1}S^{*}B_{1}^{-1}-\Sigma_{2}S^{*}B_{2}^{-1})\mathbf{v}\|_{L^{2}(\Omega)}\leq\|\Sigma_{1}S^{*}B_{1}^{-1}-\Sigma_{2}S^{*}B_{2}^{-1}\|_{\infty}\|\mathbf{v}\|
\end{equation*}
where $\|\mathbf{v}\|$ denotes the standard Euclidean norm of $\mathbf{v}$. We proceed by bounding the operator norm in the inequality above:
\begin{align*}
    \|\Sigma_{1}S^{*}B_{1}^{-1}-\Sigma_{2}S^{*}B_{2}^{-1}\|_{\infty} &= \|\Sigma_{1}S^{*}B_{1}^{-1}-\Sigma_{2}S^{*}B_{1}^{-1}+\Sigma_{2}S^{*}B_{1}^{-1}-\Sigma_{2}S^{*}B_{2}^{-1}\|_{\infty} \\
    &\leq \|(\Sigma_1-\Sigma_2)S^{*}B_{1}^{-1}\|_{\infty}+\|\Sigma_{2}S^{*}(B_{1}^{-1}-B_{2}^{-1})\|_{\infty} \\
    &\leq\|\Sigma_1-\Sigma_2\|_{\infty}\|B_{1}^{-1}\|_{\infty}+\|\Sigma_{2}\|_{\infty}\|B_{1}^{-1}-B_{2}^{-1}\|_{\infty} \\
    &\leq \|\Sigma_{1}-\Sigma_{2}\|_{1}\beta+(\|\Sigma_2-\Sigma_{1}\|_{1}+\|\Sigma_{1}\|_{\infty})(\beta^{2}\gamma^{2}h^{2}+\mathcal{O}(h^{4})) \\
    &\leq\beta(\gamma^{2}h^{2}+\mathcal{O}(h^4))+(\|\Sigma_1\|_{\infty}+\gamma^{2}h^{2}+\mathcal{O}(h^4))(\beta^{2}\gamma^{2}h^{2}+\mathcal{O}(h^4)) \\
    &=(\beta\gamma^{2}+\|\Sigma_1\|_{\infty}\beta^{2}\gamma^{2})h^{2}+\mathcal{O}(h^4) \\
    &=\gamma_{1}h^{2}+\mathcal{O}(h^4)
\end{align*}
where $\gamma_{1}:=\beta\gamma^{2}+\|\Sigma_{1}\|_{\infty}\beta^{2}\gamma^{2}>0$ is a constant independent of $h$. Note that we have utilised the fact that the operator norm is bounded above by the trace norm together with the bounds on $\|\Sigma_{1}-\Sigma_{2}\|_{1}$ and $\|B_{1}^{-1}-B_{2}^{-1}\|_{\infty}$. Thus, the second term in the bound on the norm of the difference in posterior means is bounded above as follows:
\begin{equation*}
    \|(\Sigma_{1}S^{*}B_{1}^{-1}-\Sigma_{2}S^{*}B_{2}^{-1})\mathbf{v}\|_{L^{2}(\Omega)}\leq\gamma_{1}\|\mathbf{v}\|h^{2}+\mathcal{O}(h^4)
\end{equation*}

We now proceed to obtain an upper-bound on the third and final term in the bound on the norm of the difference in means:
\begin{align*}
    \|\Sigma_{1}S^{*}B_{1}^{-1}Sm_{1}-\Sigma_{2}S^{*}B_{2}^{-1}Sm_{2}\|_{L^{2}(\Omega)} &= \|\Sigma_{1}S^{*}B_{1}^{-1}Sm_{1}-\Sigma_{1}S^{*}B_{1}^{-1}Sm_{2}+\Sigma_{1}S^{*}B_{1}^{-1}Sm_{2}-\Sigma_{2}S^{*}B_{2}^{-1}Sm_{2}\|_{L^{2}(\Omega)} \\
    &\leq\|\Sigma_{1}S^{*}B_{1}^{-1}(m_1-m_2)\|_{L^{2}(\Omega)}+\|(\Sigma_{1}S^{*}B_{1}^{-1}-\Sigma_{2}S^{*}B_{2}^{-1})Sm_{2}\|_{L^{2}(\Omega)} \\
    &\leq\|\Sigma_{1}\|_{\infty}\beta\|m_{1}-m_{2}\|_{L^{2}(\Omega)}+\|\Sigma_{1}S^{*}B_{1}^{-1}-\Sigma_{2}S^{*}B_{2}^{-1}\|_{\infty}\|m_{2}\|_{L^{2}(\Omega)} \\
    &\leq\beta\tilde{C}\|\bar{f}\|_{L^{2}(\Omega)}\|\Sigma_{1}\|_{\infty}h^{2}+(\gamma_{1}h^{2}+\mathcal{O}(h^4))(\|m_{1}\|_{L^{2}(\Omega)}+\tilde{C}h^{2}\|\bar{f}\|_{L^{2}(\Omega)}) \\
    &= (\beta\tilde{C}\|\bar{f}\|_{L^{2}(\Omega)}\|\Sigma_{1}\|_{\infty}+\|m_{1}\|_{L^{2}(\Omega)}\gamma_{1})h^{2}+\mathcal{O}(h^4) \\
    &=\gamma_{2}h^{2}+\mathcal{O}(h^4)
\end{align*}
where $\gamma_{2}:=\beta\tilde{C}\|\bar{f}\|_{L^{2}(\Omega)}\|\Sigma_{1}\|_{\infty}+\|m_{1}\|_{L^{2}(\Omega)}\gamma_{1}>0$ is a constant independent of $h$. Now, using these two bounds, together with the bound on $\|m_1-m_2\|_{L^{2}(\Omega)}$ we have:
\begin{align*}
    \|m^{(1)}_{u|\mathbf{v}}-m^{(2)}_{u|\mathbf{v}}\|_{L^{2}(\Omega)} &\leq \tilde{C}h^{2}\|\bar{f}\|_{L^{2}(\Omega)}+\gamma_{1}\|\mathbf{v}\|h^{2}+\gamma_{2}h^{2}+\mathcal{O}(h^{4}) \\
    &=(\tilde{C}\|\bar{f}\|_{L^{2}(\Omega)}+\gamma_{1}\|\mathbf{v}\|+\gamma_{2})h^{2}+\mathcal{O}(h^4) \\
    &=\tilde{\gamma}h^{2}+\mathcal{O}(h^4)
\end{align*}
where $\tilde{\gamma}:=\tilde{C}\|\bar{f}\|_{L^{2}(\Omega)}+\gamma_{1}\|\mathbf{v}\|+\gamma_{2}>0$ is a constant independent of $h$. Having dealt with the difference of the posterior means we now move onto the term involving the difference of the posterior covariances. We compute:
\begin{align*}
    \|\Sigma^{(1)}_{u|\mathbf{v}}-\Sigma^{(2)}_{u|\mathbf{v}}\|_{1} &= \|\Sigma_{1}-\Sigma_{1}S^{*}B_{1}^{-1}S\Sigma_{1}-\Sigma_{2}+\Sigma_{2}S^{*}B_{2}^{-1}S\Sigma_{2}\|_{1} \\
    &\leq\|\Sigma_{1}-\Sigma_{2}\|_{1}+\|\Sigma_{1}S^{*}B_{1}^{-1}S\Sigma_{1}-\Sigma_{2}S^{*}B_{2}^{-1}S\Sigma_{2}\|_{1}
\end{align*}
We now bound the second term above as follows:
\begin{align*}
    \|\Sigma_{1}S^{*}B_{1}^{-1}S\Sigma_{1}-\Sigma_{2}S^{*}B_{2}^{-1}S\Sigma_{2}\|_{1} &= \|\Sigma_{1}S^{*}B_{1}^{-1}S\Sigma_{1}-\Sigma_{2}S^{*}B_{2}^{-1}S\Sigma_{1}+\Sigma_{2}S^{*}B_{2}^{-1}S\Sigma_{1}-\Sigma_{2}S^{*}B_{2}^{-1}S\Sigma_{2}\|_{1} \\
    &\leq\|(\Sigma_{1}S^{*}B_{1}^{-1}-\Sigma_{2}S^{*}B_{2}^{-1})S\Sigma_{1}\|_{1}+\|\Sigma_{2}S^{*}B_{2}^{-1}S(\Sigma_1-\Sigma_2)\|_{1} \\
    &\leq\|\Sigma_{1}S^{*}B_{1}^{-1}-\Sigma_{2}S^{*}B_{2}^{-1}\|_{\infty}\|\Sigma_1\|_{1}+\|\Sigma_2\|_{\infty}\|B_{2}^{-1}\|_{\infty}\|\Sigma_1-\Sigma_2\|_{1}\\
    &\leq\gamma_{1}h^{2}\|\Sigma_1\|_{1}+(\|\Sigma_{1}\|_{\infty}+\|\Sigma_{1}-\Sigma_{2}\|_{1})(\|B_{1}^{-1}\|_{\infty}+\|B_{1}^{-1}-B_{2}^{-1}\|_{\infty})\|\Sigma_{1}-\Sigma_{2}\|_{1}+\mathcal{O}(h^{4}) \\
    &\leq\gamma_{1}\|\Sigma_1\|_{1}h^{2}+\|\Sigma_{1}\|_{\infty}\beta\gamma^{2}h^{2}+\mathcal{O}(h^4)\\
    &=(\gamma_{1}\|Sigma_{1}\|_{1}+\beta\|\Sigma_1\|_{\infty}\gamma^{2})h^{2}+\mathcal{O}(h^4)\\
    &=\gamma_{3}h^{2}+\mathcal{O}(h^4)
\end{align*}
where $\gamma_3:=\gamma_{1}\|Sigma_{1}\|_{1}+\beta\|\Sigma_1\|_{\infty}\gamma^{2}>0$ is a constant independent of $h$. Using this we can conclude:
\begin{align*}
    \|\Sigma^{(1)}_{u|\mathbf{v}}-\Sigma^{(2)}_{u|\mathbf{v}}\|_{1} &\leq \gamma^{2}h^{2}+\gamma_{3}h^{2}+\mathcal{O}(h^4) \\
    &=(\gamma^{2}+\gamma_{3})h^{2}+\mathcal{O}(h^4) \\
    &=\kappa h^2+\mathcal{O}(h^4)
\end{align*}
where $\kappa:=\gamma^{2}+\gamma_{3}>0$ is a constant independent of $h$. We can now combine the bounds on terms involving the posterior means and covariances to obtain:
\begin{align*}
    W^{2}\left(\mathcal{N}(m^{(1)}_{u|\mathbf{v}},\Sigma^{(1)}_{u|\mathbf{v}}),\mathcal{N}(m^{(2)}_{u|\mathbf{v}},\Sigma^{(2)}_{u|\mathbf{v}})\right)&\leq\kappa h^{2}+\tilde{\gamma}^{2}h^{4}+\mathcal{O}(h^{2}) \\
    &\leq\kappa h^{2}+\mathcal{O}(h^4)
\end{align*}
We can thus conclude that $W\left(\mathcal{N}(m^{(1)}_{u|\mathbf{v}},\Sigma^{(1)}_{u|\mathbf{v}}),\mathcal{N}(m^{(2)}_{u|\mathbf{v}},\Sigma^{(2)}_{u|\mathbf{v}})\right)\leq\sqrt{\kappa}h+\mathcal{O}(h^2)$ as required. \qedsymbol
