Our project is focused on providing detailed analysis of the uncertainty introduced by utilising numerical approximations for solving potentially noisy PDEs and then investigating how this propagates through to further inference, when for instance observational data is incorporated. This project lies at the intersection of the fields of data assimilation, data-centric engineering, probabilistic numerics and Bayesian inference. We now provide a brief overview of the relevant background material for the project.

It is now well established that the language of probabilistic inference can be applied to numerical problems in order to provide a more detailed notion of the uncertainty resulting from numerically approximating an intractable problem \textcolor{blue}{\cite{diaconis1988bayesian,o1992bayesian,skilling1992bayesian,hennig2015probabilistic}}. Numerical algorithms can be viewed as estimation rules for a latent, often intractable quantity given the results of tractable computations. Such algorithms can be considered to perform inference and are thus open to being analysed using the formal framework of probability theory. The field of Probabilistic Numerics (PN) \textcolor{blue}{\cite{probNumericsSite}} involves the study of so called ``probabilistic numerical methods"; these are numerical algorithms which take in a probability distribution over its inputs and gives out a probability distribution over its output. Several existing numerical methods have even recently been shown to arise from specific probabilistic models. It is worth pointing out that so far we have only been referring to problems of a deterministic nature and probability theory is used as a means of providing a notion of the uncertainty inherent in using a numerical approximation to the solution of an intractable deterministic problem. In our work we will not restrict attention to purely deterministic problems but instead will consider potentially noisy PDEs. We will then seek to analyse the problem from the viewpoint of PN.

Much work has already been undertaken in the field of PN into applications to differential equations, especially for ODEs. Classic numerical algorithms for solving initial value problems (IVPs) provide an approximate solution often defined on a grid of time points. This numerical solution is often computed iteratively by collecting information from evaluations of the vector field associated to the system of differential equations. Probabilistic numerical methods instead provide probability measures, as opposed to point estimates, over the space of possible solutions to the IVP. In the PN literature there are two main approaches to solving ODEs which we now briefly outline.

The first approach (including methods from \textcolor{blue}{\cite{chkrebtii2016bayesian,conrad2017statistical,teymur2016probabilistic,lie2019strong,abdulle2020random,teymur2018implicit}}) introduces probability measures to ODE solvers by representing the distribution of all numerically possible trajectories with a set of sample paths. The computation of these sample paths varies across these works. \textcolor{blue}{\cite{chkrebtii2016bayesian}} draws them from a (Bayesian) Gaussian process regression while \textcolor{blue}{\cite{conrad2017statistical,teymur2016probabilistic,lie2019strong,teymur2018implicit}} perturb classical estimates after an integration step with suitably scaled Gaussian noise and \textcolor{blue}{\cite{abdulle2020random}} instead perturbs the classical estimate via choosing a stochastic step size.

The second approach \textcolor{blue}{\cite{schober2014probabilistic,kersting2016active,magnani2017bayesian,schober2019probabilistic,tronarp2019probabilistic,kersting2018convergence}} recasts IVPs as \textbf{stochastic filtering problems}. This method involves assuming \textit{a priori} that the solution of the IVP and a prespecified number of its derivatives follow a Gauss-Markov process that solves a particular stochastic differential equation (SDE). The evaluations of the vector field of the IVP at numerical estimates of the true solution are then regarded as imperfect evaluations of the time derivative of the solution and are thus used as a Bayesian update for the Gauss-Markov process. This approach gives an algorithm very similar in structure to that of the Kalman filter.

Probabilistic numerical methods for PDEs are much more uncommon. However, some methods do exist, which we now briefly outline. In the book ``An Introduction to Computational Stochastic PDEs" \textcolor{blue}{\citeauthor{lord2014introduction}} analyse several different methods for dealing with elliptic PDEs with random data. In particular in chapter 9 of \textcolor{blue}{\cite{lord2014introduction}} the following (random) elliptic boundary-value problem (BVP) on a domain $D\subset\mathbb{R}^2$ is considered:
\begin{align*}
-\nabla \cdot(a(x) \nabla u(x)) &=f(x), \hspace{0.25cm} \forall x \in D \\
u(x) &=g(x), \hspace{0.25cm} \forall x \in \partial D
\end{align*}
where $\{a(x)|x\in D\}$ and $\{f(x)|x\in D\}$ are second-order random fields. \textcolor{blue}{\citeauthor{lord2014introduction}} consider several methods for dealing with such a BVP. To start they first consider a variational formulation on $D$ and show that under suitable assumptions on the diffusion coefficients there is a unique solution to the variational formulation almost surely. A Galerkin FE approximation is then established for this formulation.
The FEM is then combined with the Monte Carlo method to yield what the authors call the ``Monte Carlo Finite Element Method" (MCFEM) which can be used to estimate the expectation and variance of $u(x)$. This method essentially involves drawing \textit{iid} samples from the random fields in the BVP and then applying the FEM element to the resulting elliptic BVPs. Following this a variational formulation on $D\times\Omega$ is instead considered, where $\Omega$ is the underlying sample space for the probability space where the random fields live. The associated weak form is not a convenient starting point for Galerkin approximation as it involves taking expectations with respect to the abstract set $\Omega$ and the associated probability measure.
This leads the authors to instead consider that the noise arising from the fields comes from a finite number of random-variables (i.e. the random fields are so called \textit{finite-dimensional noise}). Doing so yields an equivalent weak form on $D\times\Gamma$ where $\Gamma$ is the range of the finite-dimensional noise. Having done this a Stochastic Galerkin FEM is developed to approximate the solution to this new weak form. Both a semi-discrete and fully-discrete version are considered (discretization can now occur in two spaces). After analysing this method the authors finally consider a stochastic collocation FEM which combines collocation on the range of the finite-dimensional noise and FEM approximations on $D$. 



\clearpage
