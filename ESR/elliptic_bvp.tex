In this section we will focus on the following standard elliptic stationary conductivity problem:
\begin{align}
    \label{standard_conduct}
    \begin{split}
        \mathcal{L}u(x) &:= -\nabla\cdot(a(x)\nabla u(x))=f(x), \hspace{0.3cm} x\in\Omega \\
        u(x) &= 0, \hspace{0.3cm} x\in\partial\Omega
    \end{split}
\end{align}
where $f$ is random and has distribution $f\sim\mathcal{N}(\bar{f},K)$ as in Section \textcolor{blue}{\ref{general_framework}}. We assume that the diffusion coefficient satisfies the following assumption taken from \textcolor{blue}{\cite{lord2014introduction}}:

\noindent \textbf{Assumption 2} (regularity of coefficients) The diffusion coefficient $a(x)$ satisfies:
\begin{equation}
    \label{regularity_of_coeff}
    0<a_{\text{min}}\leq a(x)\leq a_{\text{max}}<\infty \text{ for almost all } x\in\Omega
\end{equation}
for some real constants $a_{\text{min}},a_{\text{max}}$. In particular, $a\in L^{\infty}(\Omega)$.

We will first work out the approximate prior for a specific choice of initial prior covariance $V$. This choice will correspond to solving the weak formulation of (\ref{standard_conduct}) via a finite element method. Once we work out this approximate prior we will give some conditions on the domain $\Omega$ so that the true prior is well-defined and so that we will be able to proceed with obtaining an upper-bound on the Wasserstein distance between these two priors. For now we will assume that $\Omega$ is a bounded domain and we will proceed to work out the approximate prior which corresponds to the FEM:

\subsection{FEM prior}

Throughout this section we will take the Hilbert space of functions to be $\mathcal{H}=L^{2}(\Omega)$. As noted in Section \textcolor{blue}{\ref{general_framework}} we require that the initial prior covariance operator satisfies Assumption 1. In order to check if a candidate $V$ satisfies this assumption we must first identify a subspace of $L^{2}(\Omega)$ on which the restriction of the operator $\mathcal{L}$ in (\ref{standard_conduct}) is bounded. This can easily be done in this case, as standard energy estimates give that $\mathcal{L}$ is a bounded operator from $H^{1}_{0}(\Omega)$ to $H^{-1}(\Omega)$. Thus, it suffices to check that the choice of $V$ ensures that $u$ lies almost surely in $H_{0}^{1}(\Omega)\subset L^{2}(\Omega)$. We are now in a position to define the initial prior.

Let $V:L^{2}(\Omega)\rightarrow L^{2}(\Omega)$ be defined as follows:
\begin{equation}
    \label{FEM_prior}
    Vu(x):=\sum_{i=1}^{J}\lambda_{i}\phi_{i}(x)\int_{\Omega}\phi_{i}(y)u(y)\mathrm{d}y
\end{equation}
where the $\lambda_{i}$ are non-zero real numbers. Now the support of a Gaussian measure $\mathcal{N}(m,\Sigma)$ on a Hilbert space $\mathcal{H}$ is the space $m+\overline{\operatorname{Im}(\Sigma)}$ (see \textcolor{blue}{\cite{vakhania1975topological}} for the general case of a Gaussian measure on a Banach space) where the bar denotes closure with respect to $\mathcal{H}$ and $\operatorname{Im}(\Sigma)$ is the image of $\Sigma$. The image of $V$ can easily be seen to be contained in the span of the $\{\phi_{i}\}_{i=1}^{J}$ which itself is strictly contained in $H_{0}^{1}(\Omega)$. Thus, the support of $\mathcal{N}(0,V)$ is clearly contained in $H_{0}^{1}(\Omega)$ and so if $u$ is distributed according to this measure then $u$ lies almost surely in $H_{0}^{1}(\Omega)$. This choice of $V$ therefore satisfies Assumption 1. \vspace{10pt}

\begin{remarks}
    \textit{Remark 1.} Since $u$ lies almost surely in $H_{0}^{1}(\Omega)$ we can view $\mathcal{N}(0,V)$ as a Gaussian measure on $H_{0}^{1}(\Omega)\subset L^{2}(\Omega)$ and as such the covariance operator $V$ can be instead viewed as a function from $H^{-1}(\Omega)\rightarrow H_{0}^{1}(\Omega)$ defined by:
    \begin{equation}
        Vu := \sum_{i=1}^{J}\lambda_{i}\langle u, \phi_{i}\rangle\phi_{i} \hspace{0.25cm} \forall u\in H^{-1}(\Omega)
    \end{equation}
    where now $\langle\boldsymbol{\cdot},\boldsymbol{\cdot}\rangle$ is now the $H^{-1}(\Omega),H_{0}^{1}(\Omega)$ duality pairing.

    \noindent \textit{Remark 2.} Note that taking $u\sim\mathcal{N}(0,V)$ is equivalent to saying that $u$ has a Gaussian process distribution $u\sim\mathcal{G}\mathcal{P}(0,k)$ where the $k$ is the covariance function $k:\Omega\times\Omega\rightarrow\mathbb{R}$ given by:
    \begin{equation}
        \label{cov_function_FEM_prior}
        k(x,y):=\sum_{i=1}^{J}\lambda_{i}\phi_{i}(x)\phi_{i}(y)
    \end{equation}
    To link this with the covariance operator $V$ note that we have:
    \begin{equation}
        Vu(x)=\langle u(\boldsymbol{\cdot}), k(x,\boldsymbol{\cdot})\rangle
    \end{equation}
    This connection between the two frameworks will be useful when we move to time-dependent PDEs.
    $\mathbin{\blacklozenge}$
\end{remarks}
Now we recall the information operators. Let $\{\phi_{j}\}_{j=1}^{J}$ be a truncated set of FE basis functions. Since we have taken $\mathcal{U}=H_{0}^{1}(\Omega)$ we have that $\mathcal{U}^{\prime}=H^{-1}(\Omega)$ and thus our info operators are maps $I_{j}:H^{-1}(\Omega)\rightarrow\mathbb{R}$ given by $I_{j}\boldsymbol{\cdot}=\langle\boldsymbol{\cdot},\phi_{j}\rangle$. \vspace{10pt}

\begin{remark}
    Note that when $v\in L^{2}(\Omega)\subset H^{-1}(\Omega)$ we have, since $\phi_{j}\in H^{1}_{0}(\Omega)$, that
    \begin{equation*}
        I_{j}v=\langle v,\phi_{j}\rangle=(v,\phi_{j})_{L^2}
    \end{equation*}
    where $(\boldsymbol{\cdot},\boldsymbol{\cdot})_{L^2}$ is the standard $L^2(\Omega)$ inner product. This is essentially because of the identification of $L^2$ functions with linear functionals on $H^{1}_{0}(\Omega)$.
    $\mathbin{\blacklozenge}$
\end{remark}

It can be shown that for these information operators we have:
\begin{equation}
    \label{info_operator_with_pde_op}
    I_{j}\mathcal{L}u=\int_{\Omega}a(x)\nabla u(x)\cdot\nabla \phi_{j}(x)\mathrm{d}x
\end{equation}
In order to continue with the derivation of the approximate prior as in Section \textcolor{blue}{\ref{general_framework}} it will prove useful to express $V$ in a different form as follows. We note that we can write:
\begin{equation}
    Vu = \Phi^{*}\Lambda\left\langle u, \Phi\right\rangle
\end{equation}
where $\Phi:=(\phi_{1},\dots,\phi_{J})^{T}$ and $\Lambda:=\text{diag}\{\lambda_{i}\}_{i=1}^{J}$ is a $J\times J$ matrix. To be precise we note that the above equation means:
\begin{equation}
    Vu(x) = \Phi(x)^{*}\Lambda\left\langle u,\Phi\right\rangle
\end{equation}
where $\Phi(x)^{*}=(\phi_{1}(x),\dots,\phi_{J}(x))^{T}$ is a vector in $R^{J}$. It should also be pointed out that $\left\langle u,\Phi\right\rangle$ should be interpreted as the following column vector in $\mathbb{R}^{J}$:
\begin{equation}
    (\langle u,\phi_{1}\rangle,\dots,\langle u,\phi_{J}\rangle)^{T}=\mathcal{I}u
\end{equation}
thus we see that we can write:
\begin{equation}
    Vu=\Phi^{*}\Lambda\mathcal{I}u \hspace{0.25cm} \forall u\in H^{-1}(\Omega)
\end{equation}
i.e. $V=\Phi^{*}\Lambda\mathcal{I}$. \vspace{10pt}

\begin{remark}
    It will also prove useful later to notice that $\Phi^{*}$ is actually the same operator as $\mathcal{I}^{*}$. To see this we note that $\Phi^{*}:\mathbb{R}^{J}\rightarrow H_{0}^{1}(\Omega)$ is defined by:
    \begin{equation}
        \label{projection_into_fem_space}
        \Phi^{*}\mathbf{v}:=\sum_{i=1}^{J}v_{i}\phi_{i}
    \end{equation}
    where the vector $\mathbf{v}$ has components $v_{i}, \hspace{0.1cm} i=1,\dots,J$. Thus, $\Phi$, the adjoint of $\Phi^{*}$ is a mapping from $H^{-1}(\Omega)\rightarrow\mathbb{R}^{J}$ which is determined by the relation:
    \begin{equation}
        \langle u,\Phi^{*}\mathbf{v}\rangle = \left(\Phi u,\mathbf{v}\right) \hspace{0.25cm} \forall u\in H^{-1}(\Omega) \hspace{0.1cm} \forall \mathbf{v}\in\mathbb{R}^{J}
    \end{equation}
    where $(\boldsymbol{\cdot},\boldsymbol{\cdot})$ is the standard inner product in $\mathbb{R}^{J}$. Using this equation it is a simple matter to show that $\Phi=\mathcal{I}$ and so $\Phi^{*}=\mathcal{I}^{*}$ as claimed. We can thus also express $V$ as $V=\mathcal{I}^{*}\Lambda\mathcal{I}=\Phi^{*}\Lambda\Phi$.
    $\mathbin{\blacklozenge}$
\end{remark}

We are now in a position to work out the distribution of $u$ conditional on ``observing" $\mathcal{I}\mathcal{L}u=F$ for a fixed realistion of $f$. This distribution is given by $\mathcal{N}(a,\Sigma)$ where the definitions of $a$ and $\Sigma$ are given by equations (\ref{post_mean_before_averaging}) and (\ref{post_var_before_averaging}) respectively. We work out each of these separately. First we compute the mean:
\begin{align*}
    a &= V\mathcal{L}^{*}\mathcal{I}^{*}(\mathcal{I}\mathcal{L}V\mathcal{L}^{*}\mathcal{I}^{*})^{-1}F \\
    &= \Phi^{*}\Lambda\mathcal{I}\mathcal{L}^{*}\mathcal{I}^{*}(\mathcal{I}\mathcal{L}\Phi^{*}\Lambda\mathcal{I}\mathcal{L}^{*}\mathcal{I}^{*})^{-1}F \\
    &=\Phi^{*}\Lambda(\mathcal{I}\mathcal{L}\mathcal{I}^{*})^{*}(\mathcal{I}\mathcal{L}\Phi^{*}\Lambda(\mathcal{I}\mathcal{L}\mathcal{I}^{*})^{*})^{-1}F \\
    &=\Phi^{*}\Lambda(\mathcal{I}\mathcal{L}\Phi^{*})^{*}(\mathcal{I}\mathcal{L}\Phi^{*}\Lambda(\mathcal{I}\mathcal{L}\Phi^{*})^{*})^{-1}F
\end{align*}
where we have used that $\mathcal{I}^{*}=\Phi^{*}$.

In order to simplify the mean we will work out what $\mathcal{I}\mathcal{L}\Phi^{*}$ is. We compute:
\begin{align*}
    \mathcal{I}\mathcal{L}\Phi^{*} &= \begin{pmatrix}
                                        I_{1} \\
                                        \vdots \\
                                        I_{J}
                                      \end{pmatrix}
                                      \begin{pmatrix}
                                          \mathcal{L}\phi_{1} & \dots & \mathcal{L}\phi_{J}
                                      \end{pmatrix} \\
                                      &=: A
\end{align*}
where the $J\times J$ matrix $A$ has $ij-$\textit{th} entry given by:
\begin{align}
    A_{ij}&:=I_{i}\mathcal{L}\phi_{j} \\
          &= \int_{\Omega}a(x)\nabla\phi_{i}(x)\cdot\nabla\phi_{j}(x)\mathrm{d}x
\end{align}
where we have utilised equation (\ref{info_operator_with_pde_op}). This matrix $A$ is the standard Galerkin stiffness matrix which appears in the finite element method. This matrix is a real symmetric matrix and so $A^{*}=A$. Further, this matrix can be shown to be invertible since the diffusion coefficient $a(x)$ satisfies Assumption 2 (for details of this see the \textcolor{blue}{\nameref{appendix}}). We can thus simplify the mean as follows:
\begin{align*}
    a &= \Phi^{*}\Lambda(\mathcal{I}\mathcal{L}\Phi^{*})^{*}(\mathcal{I}\mathcal{L}\Phi^{*}\Lambda(\mathcal{I}\mathcal{L}\Phi^{*})^{*})^{-1}F \\
    &=\Phi^{*}\Lambda A^{*}(A\Lambda A^{*})^{-1}F \\
    &=\Phi^{*}\Lambda A A^{-1} \Lambda^{-1}A^{-1}F \\
    &=\Phi^{*}A^{-1}F
\end{align*}
We thus see that when expressed in terms of the finite element basis the mean function has coefficients given by the vector $\hat{a}:=A^{-1}F$ just like FEM gives. We now turn to the covariance and compute:
\begin{align*}
    \Sigma &= V-V\mathcal{L}^{*}\mathcal{I}^{*}(\mathcal{I}\mathcal{L}V\mathcal{L}^{*}\mathcal{I}^{*})^{-1}\mathcal{I}\mathcal{L}V \\
    &= V - \Phi^{*}\Lambda\mathcal{I}\mathcal{L}^{*}\mathcal{I}^{*}(A\Lambda A)^{-1}\mathcal{I}\mathcal{L}\Phi^{*}\Lambda\mathcal{I} \\
    &=V-\Phi^{*}\Lambda(\mathcal{I}\mathcal{L}\mathcal{I}^{*})^{*}A^{-1}\Lambda^{-1}A^{-1}A\Lambda\mathcal{I} \\
    &= V - \Phi^{*}\Lambda AA^{-1}\mathcal{I} \\
    &= V - \Phi^{*}\Lambda\mathcal{I} = V-V = 0
\end{align*}
We can thus see that when we choose our covariance operator $V$ as above that the posterior (for a fixed realisation of $F$) collapses to a point measure located at the FEM solution to the PDE.

We now move on to work out the marginalization over $f$. Proposition \textcolor{blue}{\ref{first_prop}} tells us that we should expect to obtain a Gaussian distribution with mean and covariance given by $Q\bar{F}$ and $\Sigma+QK_{\mathcal{I}}Q^{*}$ respectively. We have shown by the above that for the FEM case we have $Q=\Phi^{*}A^{-1}$ and $\Sigma=0$. We thus should expect to obtain the following:
\begin{equation}
    \label{approximate_fem_prior}
    \mathcal{N}(\Phi^{*}A^{-1}\bar{F},\Phi^{*}A^{-1}K_{\mathcal{I}}A^{-1}\Phi)
\end{equation}
as our averaged distribution. However, in the proof of Proposition \textcolor{blue}{\ref{first_prop}} the calculations required the existence of the inverse of $\Sigma_{N}=P\Sigma P^{*}$ which in the FEM case is the $0$ matrix since $\Sigma=0$. However, it is still possible to redo the calculation of the expectation of an arbitrary bounded cylindrical test function which we performed in Section \textcolor{blue}{\ref{general_framework}}. The details of this are left to the \textcolor{blue}{\nameref{appendix}}, but the result is the same as expected from Proposition \textcolor{blue}{\ref{first_prop}}. Thus, the approximate prior arising from this choice of $V$ is given by (\ref{approximate_fem_prior}). It should be noted that this Gaussian measure directly links to what is found at the end of section 2 in the Statistical FEM paper \textcolor{blue}{\cite{girolami2019statistical}}. It is also worth pointing out that this probabilistic numerical method for solving the BVP (\ref{standard_conduct}) gives a distribution whose mean coincides with the weak solution obtained by solving the problem with forcing given by $\bar{f}$ using FEM. Thus, this PNM recovers a classical method. We now move onto discussing the true prior.

\subsection{Prior from the true solution}

As discussed in Section \textcolor{blue}{\ref{prior_true_solution}} provided that the solution operator to the BVP is bounded we can work out that the ``true" distribution of $u$ should be:
\begin{equation}
    \label{true_prior_elliptic}
    u\sim\mathcal{N}(\mathcal{L}^{-1}\bar{f},\mathcal{L}^{-1}K(\mathcal{L}^{-1})^{*})
\end{equation}

For the BVP under consideration (\ref{standard_conduct}) the following assumptions on $\Omega$ and its boundary $\partial\Omega$ will prove to be sufficient for our purposes here:

\noindent \textbf{Assumption 3 } $\Omega$ is a bounded, convex polygonal\footnote{The polygonal assumption can easily be relaxed but we assume it here for simplicity.} domain whose boundary $\partial\Omega$ is thus a piecewise smooth curve.

Under this assumption on the domain and its boundary we have that the solution operator $\mathcal{L}^{-1}$ to the BVP is indeed bounded and so the ``true" prior can be taken to be the Gaussian (\ref{true_prior_elliptic}). We now move on to quantifying how different the approximate prior (\ref{approximate_fem_prior}) and the true prior (\ref{true_prior_elliptic}) are.

\subsection{Upperbound on the Wasserstein Distance between the two priors}

We now have two Gaussian priors for the BVP (\ref{standard_conduct}) given by $\nu_{i}=\mathcal{N}(m_{i},\Sigma_{i})$ for $i=1,2$ where:
\begin{align}
    \label{true_fem_mean}
    m_{1} &:= \mathcal{L}^{-1}\bar{f} \\
    \label{approx_fem_mean}
    m_{2} &:= \Phi^{*}A^{-1}\bar{F} \\
    \label{true_fem_cov}
    \Sigma_{1} &:= \mathcal{L}^{-1}K(\mathcal{L}^{-1})^{*} \\
    \label{approx_fem_cov}
    \Sigma_{2} &:= \Phi^{*}A^{-1}K_{\mathcal{I}}A^{-1}\Phi
\end{align}
We will now consider quantifying how close these two distributions $\nu_1,\nu_2$ are as a function of the FEM mesh size $h$. This will be achieved by obtaining an upperbound for the Wasserstein distance between $\nu_1,\nu_2$. In order to establish this upperbound a connection between the Wasserstein distance between Gaussian measures and the Procrustes Metric on covariance operators \textcolor{blue}{\cite{masarotto2019procrustes}} will be exploited. We start by first giving the definition of the Wasserstein distance between two probability measures $\mu,\nu$ on a normed space $\mathcal{H}$, $W(\mu,\nu)$:
\begin{equation}
    \label{wasserstein_def}
    W^{2}(\mu,\nu)=\inf_{\pi\in\Gamma(\mu,\nu)}\int_{\mathcal{H}\times\mathcal{H}}\|x-y\|^{2}_{\mathcal{H}}\mathrm{d}\pi(x,y)
\end{equation}
where $\Gamma(\mu,\nu)$ is the set of couplings of $\mu$ and $\nu$, i.e.
\begin{equation*}
    \Gamma(\mu,\nu):=\{\text{Borel probability measures } \pi \text{ on } \mathcal{H}\times\mathcal{H} \hspace{0.2cm} | \hspace{0.2cm} \pi(E\times\mathcal{H})=\mu(E) \text{ and } \pi(\mathcal{H}\times F)=\nu(F) \text{ for all Borel } E,F\subset\mathcal{H} \}
\end{equation*}
When $\mu,\nu$ are both Gaussian measures an explicit expression can be obtained for the Wasserstein distance. Suppose $\mu=\mathcal{N}(m_1,\Sigma_1)$ and $\nu=\mathcal{N}(m_2,\Sigma_2)$. One has \textcolor{blue}{\cite{masarotto2019procrustes}},
\begin{equation}
    W^{2}(\mu,\nu)=\|m_1-m_2\|^{2}_{\mathcal{H}}+\operatorname{tr}(\Sigma_1)+\operatorname{tr}(\Sigma_2)-2\operatorname{tr}\sqrt{\Sigma_{1}^{1/2}\Sigma_{2}\Sigma_{1}^{1/2}}
\end{equation}
This formula is true in both the finite and infinite dimensional cases. The term $\operatorname{tr}\sqrt{\Sigma_{1}^{1/2}\Sigma_{2}\Sigma_{1}^{1/2}}$ is difficult to analyse in our situation and as such we will make use of Proposition 3 from \textcolor{blue}{\cite{masarotto2019procrustes}} which states: \vspace{10pt}
\begin{proposition}
    The Procrustes distance between two trace-class operators $\Sigma_{1}$ and $\Sigma_{2}$ on $\mathcal{H}$ coincides with the Wasserstein distance between two second-order Gaussian processes $\mathcal{N}(0,\Sigma_{1})$ and $\mathcal{N}(0,\Sigma_2)$ on $\mathcal{H}$,
    $$\Pi(\Sigma_{1},\Sigma_{2}):=\inf_{R:R^{*}R=I}\|\Sigma_{1}^{1/2}-R\Sigma_{2}^{1/2}\|_{2}=W(\mathcal{N}(0,\Sigma_{1}),\mathcal{N}(0,\Sigma_{2}))$$
    where $\|\cdot\|_{2}$ is the Hilbert-Schmidt norm defined by $\|A\|_{2}=\sqrt{\operatorname{tr}(A^{*}A)}$. $\mathbin{\blacklozenge}$
\end{proposition}

Using this result one can obtain a simple upperbound on the Wasserstein distance by choosing $R=I$ in the infimum:
\begin{equation}
    W(\mathcal{N}(0,\Sigma_1),\mathcal{N}(0,\Sigma_2))=\inf_{R:R^{*}R=I}\|\Sigma_{1}^{1/2}-R\Sigma_{2}^{1/2}\|_{2} \leq \|\Sigma_{1}^{1/2}-\Sigma_{2}^{1/2}\|_{2}
\end{equation}
Now since in our case we have un-centered Gaussian measures $\nu_1,\nu_2$ we must first link the Wasserstein distance between $\nu_1,\nu_2$ to the Wasserstein distance of the centred measures $\nu_{1}^{*},\nu_{2}^{*}$ using a general result mentioned in \textcolor{blue}{\cite{cuesta1996lower}}:
\begin{equation}
    W^{2}(\nu_1,\nu_2)=\|m_1-m_2\|_{\mathcal{H}}^{2}+W^{2}(\nu_{1}^{*},\nu_{2}^{*})
\end{equation}

For our particular case, as mentioned previously, we will consider the underlying Hilbert space of functions to be $\mathcal{H}=L^2(\Omega)$. As such the norm for the difference in means is the $L^2$ norm. However, note that the solution to our BVP (\ref{standard_conduct}) lies in a subspace $H_{0}^{1}(\Omega)\subset L^{2}(\Omega)$. We have,
\begin{equation}
    W^{2}(\nu_1,\nu_2)=\|m_1-m_2\|_{L^{2}(\Omega)}^{2}+W^{2}(\mathcal{N}(0,\Sigma_1),\mathcal{N}(0,\Sigma_2))
\end{equation}
where now the $m_{i},\Sigma_{i}, i=1,2$ are given by equations (\ref{true_fem_mean}-\ref{approx_fem_cov}). We now go about obtaining an upperbound on each of these two terms. We want to control each term by the FEM mesh size $h$. For brevity we will now assume that $\Omega\subset\mathbb{R}^2$ so we have a 2-dimensional problem. The analysis follows in almost exactly the same way for $\mathbb{R}^{d}$. We will assume that $\Omega$ satisfies Assumption 3. The convexity asssumption gives us that the $H^{2}(\Omega)$ norm of the variational solution of our PDE is controlled by the $L^2$ norm of the RHS. We now take our FEM mesh to be a triangulation of $\Omega$ with $h$ being the maximum side length of any triangle in the triangulation. We require a further technical assumption:

\noindent \textbf{Assumption 4 } The meshes under consideration remain regular in the sense that as we refine the mesh by decreasing $h$ to 0 the angles of all triangles are bounded below independently of $h$.

Under all of these assumptions we have the following Theorem:
\vspace{10pt}

\begin{theorem}
    \label{prior_distance_bound}
    Let $u$ satisfy (\ref{standard_conduct}). Assuming that Assumptions 2-4 hold we have that the Wasserstein distance between the two priors $\nu_1,\nu_2$ is bounded above by $\mathcal{O}(h)$ as $h\rightarrow 0$. In particular, there exists a constant $\gamma>0$, independent of $h$, such that the following holds:
    \begin{equation}
        \label{bound_on_distance}
        W(\nu_{1},\nu_{2})\leq \gamma h +\mathcal{O}(h^2)
    \end{equation}
    Furthermore, we observe that as $h\rightarrow 0$ the approximate FEM prior $\nu_2$ converges in distribution to the true prior $\nu_1$.
\end{theorem}

\noindent \textit{Proof:} Since $m_1=\mathcal{L}^{-1}\bar{f}$ is the solution to the following elliptic BVP:
\begin{align*}
    -\nabla\cdot(a(x)\nabla v(x)) &= \bar{f}(x), \hspace{0.2cm} x\in\Omega \\
    v &= 0, \hspace{0.2cm} x\in\partial\Omega
\end{align*}
and since $m_2=\Phi^{*}A^{-1}\bar{F}=\Phi^{*}A^{-1}\mathcal{I}\bar{f}$ is the FEM solution to the variational formulation of the above problem the error analysis of FEM transfers over to allow us to bound the norm of the difference of the means as follows:
\begin{equation}
    \label{bound_on_diff_means}
    \|m_1-m_2\|_{L^{2}(\Omega)}\leq Ch^{2}\|m_{1}\|_{H^{2}(\Omega)}\leq\tilde{C}h^{2}\|\bar{f}\|_{L^{2}(\Omega)}
\end{equation}
for some constances $C,\tilde{C}>0$. We have utilised in the last inequality above the assumption that the $H^2$ norm of the true solution can be controlled by the $L^2$ norm of $\bar{f}$. The assumptions and error analysis is taken from Chapter 5 of \textcolor{blue}{\cite{larsson2008partial}} (see in particular Theorem 5.4).

We now move on to getting an upperbound for the second term. Using the link with the Procrustes distance discussed above we have:
\begin{equation}
    W^2(\mathcal{N}(0,\Sigma_1),\mathcal{N}(0,\Sigma_2))\leq\|\Sigma_{1}^{1/2}-\Sigma_{2}^{1/2}\|_{2}^{2}
\end{equation}

The RHS of the above is still difficult to deal with so we make use of Lemma 4.1 from \textcolor{blue}{\cite{powers1970free}} to obtain:
\begin{equation}
    W^2(\mathcal{N}(0,\Sigma_1),\mathcal{N}(0,\Sigma_2))\leq\|\Sigma_{1}^{1/2}-\Sigma_{2}^{1/2}\|_{2}^{2}\leq\|\Sigma_1-\Sigma_2\|_{1}
\end{equation}
where $\|\cdot\|_1$ is the trace norm or nuclear norm defined by $\|A\|_{1}=\operatorname{tr}(\sqrt{A^{*}A})$.

We now investigate this term:
\begin{align*}
    \|\Sigma_{1}-\Sigma_2\|_{1}&=\|\mathcal{L}^{-1}K(\mathcal{L}^{-1})^{*}-\Phi^{*}A^{-1}K_{\mathcal{I}}A^{-1}\Phi\|_{1} \\
    &=\|\mathcal{L}^{-1}K(\mathcal{L}^{-1})^{*}-\Phi^{*}A^{-1}\mathcal{I}K\mathcal{I}^{*}A^{-1}\Phi\|_{1} \\
    &=\|\mathcal{L}^{-1}K(\mathcal{L}^{-1})^{*}-\Phi^{*}A^{-1}\mathcal{I}K(\Phi^{*}A^{-1}\mathcal{I})^{*}\|_{1}
\end{align*}
where we have used the defintion of $K_{\mathcal{I}}$ and the fact that $A$ and hence $A^{-1}$ is self-adjoint. From this simplification we can see that we can control how ``close" the two covariance operators are because we can control how ``close" $\Phi^{*}A^{-1}\mathcal{I}$ is to $\mathcal{L}^{-1}$. To be more precise:
\begin{align*}
    \|\Sigma_{1}-\Sigma_{2}\|_{1}&=\|\mathcal{L}^{-1}K(\mathcal{L}^{-1})^{*}-\Phi^{*}A^{-1}\mathcal{I}K(\Phi^{*}A^{-1}\mathcal{I})^{*}\|_{1} \\
    &=\|\mathcal{L}^{-1}K(\mathcal{L}^{-1})^{*}-\Phi^{*}A^{-1}\mathcal{I}K(\mathcal{L}^{-1})^{*}+\Phi^{*}A^{-1}\mathcal{I}K(\mathcal{L}^{-1})^{*}-\Phi^{*}A^{-1}\mathcal{I}K(\Phi^{*}A^{-1}\mathcal{I})^{*}\|_{1} \\
    &\leq\|\mathcal{L}^{-1}K(\mathcal{L}^{-1})^{*}-\Phi^{*}A^{-1}\mathcal{I}K(\mathcal{L}^{-1})^{*}\|_{1} + \|\Phi^{*}A^{-1}\mathcal{I}K(\mathcal{L}^{-1})^{*}-\Phi^{*}A^{-1}\mathcal{I}K(\Phi^{*}A^{-1}\mathcal{I})^{*}\|_{1} \\
    &= \|(\mathcal{L}^{-1}-\Phi^{*}A^{-1}\mathcal{I})K(\mathcal{L}^{-1})^{*}\|_{1} + \|\Phi^{*}A^{-1}\mathcal{I}K(\mathcal{L}^{-1}-\Phi^{*}A^{-1}\mathcal{I})^{*}\|_{1} \\
    &\leq\|\mathcal{L}^{-1}-\Phi^{*}A^{-1}\mathcal{I}\|_{\infty}\|K\mathcal{L}^{-1}\|_{1}+\|\Phi^{*}A^{-1}\mathcal{I}K\|_{1}\|(\mathcal{L}^{-1}-\Phi^{*}A^{-1}\mathcal{I})^{*}\|_{\infty} \\
    &\leq\|\mathcal{L}^{-1}-\Phi^{*}A^{-1}\mathcal{I}\|_{\infty}(\|K\|_{1}\|\mathcal{L}^{-1}\|_{\infty}+\|\Phi^{*}A^{-1}\mathcal{I}\|_{\infty}\|K\|_{1})
\end{align*}
where we have utilized Holder's inequality and the sub-multiplicativity of Schatten-p norms (\textit{note: the} $\|\cdot\|_{\infty}$ \textit{is the operator norm here and is a special case of the Schatten-p norm for p being infinity}). As mentioned, we can control how ``close" $\Phi^{*}A^{-1}\mathcal{I}$ is to $\mathcal{L}^{-1}$. To do this we note that (\ref{bound_on_diff_means}) holds for all $\bar{f}$ in $L^{2}(\Omega)$, i.e. we have:
\begin{equation*}
    \|m_1-m_2\|_{L^{2}(\Omega)}=\|(\mathcal{L}^{-1}-\Phi^{*}A^{-1}\mathcal{I})\bar{f}\|_{L^{2}(\Omega)}\leq \tilde{C}h^{2}\|\bar{f}\|_{L^2(\Omega)} \hspace{0.25cm} \forall\bar{f}\in L^{2}(\Omega)
\end{equation*}

This implies that we can bound the operator norm\footnote{Note that here we are viewing both $\mathcal{L}^{-1}$ and $\Phi^{*}A^{-1}\mathcal{I}$ as operators from $L^{2}(\Omega)$ to itself. They are however more generally operators from $H^{-1}(\Omega)$ to $H^{1}_{0}(\Omega)$. We will consider here the case that $\bar{f}$ is in $L^{2}(\Omega)$ but this assumption can be relaxed.} of $\mathcal{L}^{-1}-\Phi^{*}A^{-1}\mathcal{I}$ by $\tilde{C}h^{2}$, i.e.,
\begin{equation}
    \label{bound_on_diff_of_solution_operators}
    \|\mathcal{L}^{-1}-\Phi^{*}A^{-1}\mathcal{I}\|_{\infty}\leq \tilde{C}h^{2}
\end{equation}

Utilising this upper bound together with the fact that this implies that $\|\Phi^{*}A^{-1}\mathcal{I}\|_{\infty}$ is bounded by $\|\mathcal{L}^{-1}\|_{\infty}+\tilde{C}h^2$ we have:
\begin{equation}
    \label{bound_on_diff_cov_operators}
    W^{2}(\mathcal{N}(0,\Sigma_1),\mathcal{N}(0,\Sigma_2))\leq\|\Sigma_{1}-\Sigma_{2}\|_{1}\leq \tilde{C}h^{2}\|K\|_{1}\left(2\|\mathcal{L}^{-1}\|_{\infty}+\tilde{C}h^2\right)
\end{equation}

Combining (\ref{bound_on_diff_means}) and (\ref{bound_on_diff_cov_operators}) we now have:
\begin{align*}
    W^{2}(\nu_{1},\nu_{2})&=\|m_1-m_2\|_{L^{2}(\Omega)}^{2}+W^{2}(\mathcal{N}(0,\Sigma_1),\mathcal{N}(0,\Sigma_2)) \\
    &\leq \tilde{C}^{2}h^4\|\bar{f}\|^{2}_{L^{2}(\Omega)}+ \tilde{C}h^{2}\|K\|_{1}\left(2\|\mathcal{L}^{-1}\|_{\infty}+\tilde{C}h^2\right) \\
    &=2\tilde{C}h^{2}\|K\|_{1}\|\mathcal{L}^{-1}\|_{\infty}+\mathcal{O}(h^4)
\end{align*}

We thus have an upper bound on the Wasserstein distance between $\nu_{1},\nu_{2}$ in terms of the FEM mesh size $h$:
\begin{equation}
    W(\nu_{1},\nu_{2})\leq h \sqrt{2\tilde{C}\|K\|_{1}\|\mathcal{L}^{-1}\|_{\infty}+\mathcal{O}(h^2)}\leq \gamma h +\mathcal{O}(h^2)
\end{equation}
where $\gamma:=\sqrt{2\tilde{C}\|K\|_{1}\|\mathcal{L}^{-1}\|_{\infty}}>0$ is a constant. Note that we have also used that both\footnote{$K$ is trace-class since it is a covariance operator and thus its trace-norm is bounded.} $\|K\|_{1}$ and $\|\mathcal{L}^{-1}\|_{\infty}$ are bounded and independent of $h$. We also see that as $h\rightarrow 0$ the Wasserstein distance between $\nu_1,\nu_2$ goes to 0 and so the approximate prior converges (in distribution) to the true prior as $h\rightarrow 0$ (see Proposition 4 in \textcolor{blue}{\cite{masarotto2019procrustes}}).
\qedsymbol

Having obtained this upperbound on the distance between the true priors we are now in a position to investigate how this ``propagates" forwards to further inference. In particular, we will now look at incorporating sensor data and seeing how different the two resulting posteriors are from using the two different priors $\nu_1,\nu_2$ above.

\subsection{Upperbound on the Wasserstein Distance between the two posteriors}

As discussed in Section \textcolor{blue}{\ref{incorporating_obs_data}} we will now assume that we have noisy observations of the value of $u$ at some points $y_1,\dots,y_s\in\Omega$ coming from a sensor. We wish to update our belief in the distribution of $u$ using this sensor data, and as discussed either of the $\nu_i$ can be used as prior distributions. We thus obtain two different posteriors and these are given by $u|\mathbf{v}\sim\mathcal{N}(m^{(i)}_{u|\mathbf{v}},\Sigma^{(i)}_{u|\mathbf{v}})$ where $m^{(i)}_{u|\mathbf{v}}:=m_{i}+\Sigma_{i}S^{*}B_{i}^{-1}(\mathbf{v}-Sm_{i})$ and $\Sigma^{(i)}_{u|\mathbf{v}}:=\Sigma_{i}-\Sigma_{i}S^{*}B_{i}^{-1}S\Sigma_{i}$ as shown in Section \textcolor{blue}{\ref{incorporating_obs_data}}. We have denoted $B_{i}:=\epsilon^{2}I+S\Sigma_{i}S^{*}$ for convenience. We can no go about obtaining an upperbound for the Wasserstein distance between these two posteriors. The result is given in the following Theorem and we outline a proof of this.
\vspace{10pt}

\begin{theorem}
    \label{posterior_difference_bound}
    Let $u$ satisfy (\ref{standard_conduct}). Assuming that Assumptions 2-4 hold we have that the Wasserstein distance between the two posteriors is bounded above by $\mathcal{O}(h)$ as $h\rightarrow 0$. In particular, there exists a constant $\kappa>0$, independent of $h$, such that the following holds:
    \begin{equation}
        \label{post_bound}
        W\left(\mathcal{N}(m^{(1)}_{u|\mathbf{v}},\Sigma^{(1)}_{u|\mathbf{v}}),\mathcal{N}(m^{(2)}_{u|\mathbf{v}},\Sigma^{(2)}_{u|\mathbf{v}})\right)\leq\sqrt{\kappa} h + \mathcal{O}(h^2)
    \end{equation}
    Furthermore, we observe that as $h\rightarrow 0$ the second posterior obtained using the prior $\nu_2$ converges in distribution to the first posterior obtained using the prior $\nu_1$.
\end{theorem}

\noindent \textit{Proof:} We have,
\begin{align*}
    W^{2}\left(\mathcal{N}(m^{(1)}_{u|\mathbf{v}},\Sigma^{(1)}_{u|\mathbf{v}}),\mathcal{N}(m^{(2)}_{u|\mathbf{v}},\Sigma^{(2)}_{u|\mathbf{v}})\right)&=\|m^{(1)}_{u|\mathbf{v}}-m^{(2)}_{u|\mathbf{v}}\|_{L^{2}(\Omega)}^{2}+W^{2}\left(\mathcal{N}(0,\Sigma^{(1)}_{u|\mathbf{v}}),\mathcal{N}(0,\Sigma^{(2)}_{u|\mathbf{v}})\right) \\
    &\leq \|m^{(1)}_{u|\mathbf{v}}-m^{(2)}_{u|\mathbf{v}}\|_{L^{2}(\Omega)}^{2} + \|\Sigma^{(1)}_{u|\mathbf{v}}-\Sigma^{(2)}_{u|\mathbf{v}}\|_{1}
\end{align*}
In order to obtain on upperbound on each of these terms it will help to first figure out an upperbound on $\|B_{1}^{-1}-B_{2}^{-1}\|_{\infty}$. To this end we first compute:
\begin{align*}
    \|B_{1}-B_{2}\|_{\infty}&=\|S(\Sigma_{1}-\Sigma_{2})S^{*}\|_{\infty} \\
    &\leq\|S\|_{\infty}^{2}\|\Sigma_{1}-\Sigma_{2}\|_{\infty} \\
    &\leq\|\Sigma_{1}-\Sigma_{2}\|_{1} \\
    &\leq \gamma^{2}h^{2}+\mathcal{O}(h^4)
\end{align*}
where we have used the fact that $\|S\|_{\infty}\leq 1$ since it is a projection. Since $B_{1}$ (and $B_{2}$) are invertible and for sufficiently small\footnote{It suffices to take $h<h_{*}$ where $h_{*}$ satisfies $\gamma^{2}h_{*}^{2}+\tilde{C}^{2}h_{*}^{4}<1/\|B_{1}^{-1}\|_{\infty}$.} $h$ we have that $\|B_{1}-B_{2}\|_{\infty}<1/\|B_{1}^{-1}\|_{\infty}$ we can utilise Corollary 8.2 of \textcolor{blue}{\cite{gohberg2012basic}} to deduce that:
\begin{equation}
    \label{bound_on_diff_inverses}
    \|B_{1}^{-1}-B_{2}^{-1}\|_{\infty}\leq\frac{\|B_{1}^{-1}\|_{\infty}^{2}\|B_{1}-B_{2}\|_{\infty}}{1-\|B_{1}^{-1}\|_{\infty}\|B_1-B_{2}\|_{\infty}}\leq\|B_{1}^{-1}\|_{\infty}^{2}\gamma^{2}h^{2}+\mathcal{O}(h^4)
\end{equation}
Note that $\|B_{1}^{-1}\|_{\infty}$ is a bounded constant independent of $h$ (but dependent on $\epsilon$).

We are now in a good position to derive an upper bound on the Wasserstein distance between the two posteriors. We first focus on the difference of the two means:
\begin{align*}
    \|m^{(1)}_{u|\mathbf{v}}-m^{(2)}_{u|\mathbf{v}}\|_{L^{2}(\Omega)}&=\|m_1+\Sigma_{1}S^{*}B_{1}^{-1}(\mathbf{v}-Sm_{1})-m_2-\Sigma_{2}S^{*}B_{2}^{-1}(\mathbf{v}-Sm_{2})\|_{L^{2}(\Omega)} \\
    &=\|m_1-m_2+\Sigma_{1}S^{*}B_{1}^{-1}\mathbf{v}-\Sigma_{1}S^{*}B_{1}^{-1}Sm_1-\Sigma_{2}S^{*}B_{2}^{-1}\mathbf{v}+\Sigma_{2}S^{*}B_{2}^{-1}Sm_{2}\|_{L^{2}(\Omega)} \\
    &\leq\|m_1-m_2\|_{L^{2}(\Omega)}+\|(\Sigma_{1}S^{*}B_{1}^{-1}-\Sigma_{2}S^{*}B_{2}^{-1})\mathbf{v}\|_{L^{2}(\Omega)}+\|\Sigma_{1}S^{*}B_{1}^{-1}Sm_1-\Sigma_{2}S^{*}B_{2}^{-1}Sm_2\|_{L^{2}(\Omega)}
\end{align*}

We now focus on these three terms separately. For the first term we already have from (\ref{bound_on_diff_means}) that
\begin{equation*}
    \|m_{1}-m_{2}\|_{L^{2}(\Omega)}\leq\tilde{C}h^{2}\|\bar{f}\|_{L^2(\Omega)}
\end{equation*}
The other two terms can be bounded above by utilising the bounds on $\|\Sigma_1-\Sigma_{2}\|_{1}$ and $\|B_{1}^{-1}-B_{2}^{-1}\|_{\infty}$. Details of these calculations are left to the \textcolor{blue}{\nameref{appendix}}. Upon performing these calculations we obtain:
\begin{equation}
    \|m^{(1)}_{u|\mathbf{v}}-m^{(2)}_{u|\mathbf{v}}\|_{L^{2}(\Omega)}\leq\tilde{\gamma}h^{2}+\mathcal{O}(h^4)
\end{equation}
where $\tilde{\gamma}>0$ is a constant independent of $h$. Similarly (see \textcolor{blue}{\nameref{appendix}} for details) one can obtain the following upperbound for the difference of the covariance operators:
\begin{equation}
    \|\Sigma^{(1)}_{u|\mathbf{v}}-\Sigma^{(2)}_{u|\mathbf{v}}\|_{1}\leq \kappa h^{2}+\mathcal{O}(h^4)
\end{equation}
where $\kappa>0$ is a constant independent of $h$. Combining these two upper bounds we obtain,
\begin{equation}
    W^{2}\left(\mathcal{N}(m^{(1)}_{u|\mathbf{v}},\Sigma^{(1)}_{u|\mathbf{v}}),\mathcal{N}(m^{(2)}_{u|\mathbf{v}},\Sigma^{(2)}_{u|\mathbf{v}})\right)\leq\kappa h^{2} + \mathcal{O}(h^4)
\end{equation}
and so the Wasserstein distance between the posteriors is bounded above by $\mathcal{O}(h)$ and thus goes to $0$ as $h\rightarrow 0$. Thus, the posterior obtained using the approximate prior converges in distribution to the posterior obtained using the true prior as the mesh size goes to 0.
\qedsymbol
